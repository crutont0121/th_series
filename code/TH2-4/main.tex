%%%%%%%%%%%%%%%%%%%%%%%%%%
%%% author : Yamada. T %%%
%%% made for TH series %%%
%%%%%%%%%%%%%%%%%%%%%%%%%%

\documentclass[b5paper,10pt,fleqn] {ltjsarticle}

\usepackage[margin=10truemm]{geometry}

\usepackage{pict2e, graphicx}
\usepackage{tikz}
\usetikzlibrary{intersections,calc,arrows.meta}

\usepackage{amsmath, amssymb, amsthm}
\usepackage{ascmac}
\usepackage{comment}
\usepackage{empheq}
\usepackage[shortlabels,inline]{enumitem}
\usepackage{fancybox}
\usepackage{fancyhdr}
\usepackage{here}
\usepackage{lastpage}
\usepackage{listings, jvlisting}
\usepackage{fixdif}

\usepackage{stmaryrd}
\usepackage[listings]{tcolorbox}
%\usepackage{ascolorbox}
\usepackage{titlesec}
\usepackage{ulem}
\usepackage{url}
\usepackage{verbatim}
\usepackage{wrapfig}
\usepackage{xcolor}
\usepackage{luatexja-ruby}
\usepackage{varwidth}
\usepackage[version=3]{mhchem}
\usepackage{wrapfig}


\usepackage{physics2}
	\usephysicsmodule{ab}
	\usephysicsmodule{ab.braket}
	\usephysicsmodule{ab.legacy}
	%\usephysicsmodule{braket}
	\usephysicsmodule{diagmat}
	\usephysicsmodule{xmat}
	\usephysicsmodule{nabla.legacy}
	\usephysicsmodule{qtext.legacy}

\usepackage[ISO]{diffcoeff}
\difdef { f, s } { D }
{ op-symbol = \mathrm{D} }


\newcommand{\mctext}[1]{\mbox{\textcircled{\scriptsize{#1}}}}
\newcommand{\ctext}[1]{\textcircled{\scriptsize{#1}}}
\newcommand{\ds}{\displaystyle}
\newcommand{\comb}[2]{{}_{#1}\mathrm{C}_{#2}}
\newcommand{\hs}{\hspace}
\newcommand{\vs}{\vspace}
\newcommand{\emphvs}{\vspace{1em}\notag\\}
\newcommand{\ora}{\overrightarrow}
\newcommand{\oramr}[1]{\overrightarrow{\mathrm{#1}}}
\newcommand{\tri}{\triangle}
\newcommand{\mr}{\mathrm}
\newcommand{\mb}{\mathbb}
\newcommand{\mrvec}[1]{\overrightarrow{\mathrm{#1}}}
\newcommand{\itvec}{\overrightarrow}
\newcommand{\bs}{\boldsymbol}
\newcommand{\ra}{\rightarrow}
\newcommand{\Ra}{\Rightarrow}
\newcommand{\lra}{\longrightarrow}
\newcommand{\Lra}{\Longrightarrow}
\newcommand{\la}{\leftarrow}
\newcommand{\La}{\Leftarrow}
\newcommand{\lla}{\longleftarrow}
\newcommand{\Lla}{\Longleftarrow}
\newcommand{\lr}{\leftrightarrow}
\newcommand{\llr}{\longleftrightarrow}
\newcommand{\Llr}{\Longleftrightarrow}
\renewcommand{\deg}{{}^\circ}
\newcommand{\phbox}{\fbox{\phantom{1\hspace{2em}}}}
\newcommand{\boxnum}[1]{\fbox{\phantom{\hspace{1em}}({#1})\phantom{\hspace{1em}}}}
\newcommand{\boxkana}[1]{\fbox{\phantom{\hspace{1em}}{#1}\phantom{\hspace{1em}}}}
\newcommand{\boxkm}[2]{\fbox{\, {#1}\phantom{\hspace{0.2em}} \,  ${#2}$}}
\newcommand{\hzw}{\hspace{1\zw}}

\renewcommand{\baselinestretch}{1.25}
\parindent=1\zw

%%TH2-5
\begin{document}
\noindent
\fbox{NewTH2-4} [東京大]

右図のように,断熱壁で囲まれた同一形状のシリンダーA,Bが,コックCのついた体積の無視できる細い管でつながれている.最初,コックCは閉じていて,シリンダーAには,圧力$P_0$,体積$V_0$,物質量$n$の単原子分子の理想気体が質量$m$の断熱板で閉じ込められている.
断熱板はすべり落ちないように,下からストッパーで支えられており,天井から質量の無視できるばね定数$k$のばねが取り付けられている.ばねの長さは自然長に等しい.また,シリンダーA内にはヒーターがあり,スイッチをいれると,気体を加熱することができる.シリンダーBは真空になっていて,内部の容積が$V_0$になるような高さに断熱板があり,留め具によって固定されている.断熱板の断面積を$S$,重力加速度の大きさを$g$,気体定数を$R$とする.シリンダー外部の圧力による影響は無視してよい.

\begin{enumerate}[label=\textbf{問\arabic*}]
  \item {\hzw}コックCをゆっくり開く.十分に時間が経過して,気体がシリンダーA,Bの内部に一様に充満したときの気体の状態を$\mr{Z}_1$とし,そのときの温度$T_1$と圧力$P_1$を求めよ.ただし,シリンダーA内の断熱板はストッパーから離れないものとする.
  \item {\hzw}状態$\mr{Z}_1$において,ヒーターのスイッチを入れて気体をゆっくりと加熱すると,しばらくして,シリンダーAの断熱板が動き始めた.その瞬間に,ヒーターのスイッチを切った.スイッチを切った後の気体の状態を$\mr{Z}_2$とし,そのときの気体の圧力$P_2$と温度$T_2$を求めよ.
  \item {\hzw}状態$\mr{Z}_2$において,ヒーターのスイッチを入れて気体を徐々に加熱すると,シリンダーAの断熱板がゆっくりと上方に動いた.気体の体積が$\varDelta V$だけ増えたとき,ヒーターのスイッチを切った.スイッチを切った後の気体の状態を$\mr{Z}_3$とし,状態$\mr{Z}_2$から$\mr{Z}_3$への変化に関して,以下の問いに答えよ.
  \begin{enumerate}[(1)]
    \item {\hzw}気体の圧力変化$\varDelta P$を$\varDelta V$を用いて表せ.
    \item {\hzw}期待がした仕事$W_\mr{g}$を$P_2$,$\varDelta P$,$\varDelta V$を用いて表せ.
    \item {\hzw}ヒーターが気体に与えた熱$Q_\mr{h}$を$P_2$,$V_0$,$\varDelta V$,$\varDelta P$を用いて表せ.
  \end{enumerate}
  \item {\hzw}状態$\mr{Z}_3$において,コックCを閉め,シリンダーBの断熱板の留め具を外し,その断熱板を機械的に速く上下振動させた後に,元の位置に戻し,再び,留め具で固定した.この間に,気体がなされた仕事を$W_\mr{m} \, (> 0)$とする.その後,十分に時間が経過したときの状態を$\mr{Z}_4$とする.状態$\mr{Z}_4$の温度$T_4$を$T_2$,$W_\mr{m}$を用いて表せ.
  \item {\hzw}状態$\mr{Z}_4$において,コックCをゆっくりと開くと,シリンダーAの断熱板がゆっくりと上下に動き,状態$\mr{Z}_3$と同じ状態になった.このとき,$\mr{W}_\mr{m}$と$Q_\mr{h}$の関係を記せ.また,その関係が成り立つ理由を簡潔に述べよ.
\end{enumerate}
\end{document}