%%%%%%%%%%%%%%%%%%%%%%%%%%
%%% author : Yamada. T %%%
%%% made for TH series %%%
%%%%%%%%%%%%%%%%%%%%%%%%%%

\documentclass[b5paper,10pt,fleqn] {ltjsarticle}

\usepackage[margin=10truemm]{geometry}

\usepackage{pict2e, graphicx}
\usepackage{tikz}
\usetikzlibrary{intersections,calc,arrows.meta}

\usepackage{amsmath, amssymb, amsthm}
\usepackage{ascmac}
\usepackage{comment}
\usepackage{empheq}
\usepackage[shortlabels,inline]{enumitem}
\usepackage{fancybox}
\usepackage{fancyhdr}
\usepackage{here}
\usepackage{lastpage}
\usepackage{listings, jvlisting}
\usepackage{fixdif}

\usepackage{stmaryrd}
\usepackage[listings]{tcolorbox}
%\usepackage{ascolorbox}
\usepackage{titlesec}
\usepackage{ulem}
\usepackage{url}
\usepackage{verbatim}
\usepackage{wrapfig}
\usepackage{xcolor}
\usepackage{luatexja-ruby}
\usepackage{varwidth}
\usepackage[version=3]{mhchem}
\usepackage{wrapfig}


\usepackage{physics2}
	\usephysicsmodule{ab}
	\usephysicsmodule{ab.braket}
	\usephysicsmodule{ab.legacy}
	%\usephysicsmodule{braket}
	\usephysicsmodule{diagmat}
	\usephysicsmodule{xmat}
	\usephysicsmodule{nabla.legacy}
	\usephysicsmodule{qtext.legacy}

\usepackage[ISO]{diffcoeff}
\difdef { f, s } { D }
{ op-symbol = \mathrm{D} }


\newcommand{\mctext}[1]{\mbox{\textcircled{\scriptsize{#1}}}}
\newcommand{\ctext}[1]{\textcircled{\scriptsize{#1}}}
\newcommand{\ds}{\displaystyle}
\newcommand{\comb}[2]{{}_{#1}\mathrm{C}_{#2}}
\newcommand{\hs}{\hspace}
\newcommand{\vs}{\vspace}
\newcommand{\emphvs}{\vspace{1em}\notag\\}
\newcommand{\ora}{\overrightarrow}
\newcommand{\oramr}[1]{\overrightarrow{\mathrm{#1}}}
\newcommand{\tri}{\triangle}
\newcommand{\mr}{\mathrm}
\newcommand{\mb}{\mathbb}
\newcommand{\mrvec}[1]{\overrightarrow{\mathrm{#1}}}
\newcommand{\itvec}{\overrightarrow}
\newcommand{\bs}{\boldsymbol}
\newcommand{\ra}{\rightarrow}
\newcommand{\Ra}{\Rightarrow}
\newcommand{\lra}{\longrightarrow}
\newcommand{\Lra}{\Longrightarrow}
\newcommand{\la}{\leftarrow}
\newcommand{\La}{\Leftarrow}
\newcommand{\lla}{\longleftarrow}
\newcommand{\Lla}{\Longleftarrow}
\newcommand{\lr}{\leftrightarrow}
\newcommand{\llr}{\longleftrightarrow}
\newcommand{\Llr}{\Longleftrightarrow}
\renewcommand{\deg}{{}^\circ}
\newcommand{\phbox}{\fbox{\phantom{1\hspace{2em}}}}
\newcommand{\boxnum}[1]{\fbox{\phantom{\hspace{1em}}({#1})\phantom{\hspace{1em}}}}
\newcommand{\boxkana}[1]{\fbox{\phantom{\hspace{1em}}{#1}\phantom{\hspace{1em}}}}
\newcommand{\boxkm}[2]{\fbox{\, {#1}\phantom{\hspace{0.2em}} \,  ${#2}$}}
\newcommand{\hzw}{\hspace{1\zw}}

\renewcommand{\baselinestretch}{1.25}
\parindent=1\zw


%% TH2-16
%% prob: 入試120

\begin{document}
\noindent\fbox{NewTH2-16} [東京大]

常温の水は液体(以後,単に水という)と気体(水蒸気)の2つの状態をとることができる.
どちらの状態をとるかは温度と圧力により,図1に示すように定まる.
例えば,水をシリンダーに密封して温度を30${}^\circ\mr{C}$,圧力を7000 Paにしたときは水であり,
熱を与えて,温度や圧力を多少変えても全部が水のままである.
一方,同じ30${}^\circ\mr{C}$で,圧力を1000 Paにしたときはすべて水蒸気である.
ただし,図1のB点,C点のような境界線(共存線)上の温度と圧力の時は
水と水蒸気が共存できる.
逆に,水と水蒸気が共存しているときの温度と圧力はこの境界線上の値をもつ.
温度を与えた時に定まる共存時の圧力を,その温度での蒸気圧という.
一定の圧力で共存している水と水蒸気に熱を与えると,
温度は変わらずに,熱に比例する量の水が水蒸気に変わり,全体の体積は膨張する.
1 molの水を水蒸気に変化させるために必要なエネルギーを蒸発熱とよぶ.

このことを参考にして,図2--1に示す装置のはたらきを調べよう.
断面積$A$〔$\mr{m^2}$〕で下端を閉じたシリンダーを鉛直に立てて,物質量$n$〔mol〕の水を入れ,
質量$m_1$〔kg〕のピストンで密閉し,その上に質量$m_2$〔kg〕のおもりをのせる.
シリンダーの上端を閉じてなめらかに動くことができるが,シリンダーの上方にはストッパーがついていて,ピストンの下面の高さが$L$〔m〕になるところまでしか上昇しないようになっている.
シリンダーの底にはヒーターが置かれていて,
外部からの電流でジュール熱を発生できるようになっている.
以下の過程を通じて,各瞬間の水と水蒸気の温度はシリンダー内の位置によらず等しいものとする.
また,圧力の位置による違いは無視する.

\begin{enumerate}[(1)]  
  \item 20${}^\circ\mr{C}$での蒸気圧を$p_1$〔Pa〕,30${}^\circ\mr{C}$での蒸気圧を$p_2$〔Pa〕とする.
  ピストンのみでおもりをのせないときに内部の圧力が$p_1$で,ピストンにおもりをのせたときに$p_2$になるようにしたい.$m_1$と$m_2$を求めよ.重力加速度の大きさを$g$〔$\mr{m}/\mr{s}^2$〕とする.
\item 圧力$p_2$での20${}^\circ\mr{C}$の水のモル体積(1 molあたりの体積)を$v_1$〔$\mr{m}^3 / \mr{mol}$〕とする.
  この温度でおもりをのせた状態でのシリンダー内の水の深さ$d$〔m〕を求めよ.
  なお,ヒーターの体積は無視できる.
\item 装置全体を断熱材で覆い,ピストンにおもりをのせたまま,初め20${}^\circ\mr{C}$であった水を
  ヒーターでゆっくりと30${}^\circ\mr{C}$になるまで加熱する.
  このとき,水の状態は図1のA点からB点に移る.
  20${}^\circ\mr{C}$から30${}^\circ\mr{C}$までの水の定圧モル比熱は温度によらず,$c$〔$\mr{J} / (\mr{mol}\cdot\mr{K})$〕であるとする.水を30${}^\circ\mr{C}$にするためにヒーターで発生させるジュール熱$Q_1$〔J〕を求めよ.
  なお,シリンダー,ピストン,おもり,断熱材など,水以外の物体の熱容量は無視できるものとする.
\item 30${}^\circ\mr{C}$での水をさらにヒーターでゆっくりと加熱する.
  このときの温度と圧力はB点にとどまり,水は少しずつ水蒸気に変化していく.
  図2--2のようにピストンがストッパーに達したときにも水が残っていた.B点での水のモル体積$v_2$〔$\mr{m^3}/\mr{mol}$〕とB点での水蒸気のモル体積$v_3$〔$\mr{m^3}/\mr{mol}$〕を用いて,このときの水蒸気の物質量$x$〔mol〕を求めよ.
\item 30${}^\circ\mr{C}$の水を,その温度での蒸気圧のもとで,水蒸気にするために必要となる蒸発熱を$q$〔J/mol〕とする.
  (4)の過程で,ピストンがストッパーに達するまでに,ヒーターで発生させるジュール熱$Q_2$\nolinebreak〔J〕を求めよ.
\item ピストンがストッパーに達したときにヒーターを切り,おもりを横にずらして,ストッパーにのせる.
  次に周りの断熱材を取り除き,18${}^\circ\mr{C}$の室内で装置全体がゆっくりと冷えるのを待つ.
  \begin{enumerate}[label=\ctext{\arabic*}]  
    \item 時間の経過(温度の低下)とともに,圧力がどのように変化するか述べよ.
    \item 時間の経過(温度の低下)とともに,ピストンはストッパーに接した位置と水面に接した位置の間でどのように動くか.
      動く場合にはその速さ(瞬間的か,ゆっくりか)を含めて述べよ.
  \end{enumerate}
\end{enumerate}
\end{document}
