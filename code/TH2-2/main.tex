%%%%%%%%%%%%%%%%%%%%%%%%%%
%%% author : Yamada. T %%%
%%% made for TH series %%%
%%%%%%%%%%%%%%%%%%%%%%%%%%

\documentclass[b5paper,10pt,fleqn] {ltjsarticle}

\usepackage[margin=10truemm]{geometry}

\usepackage{pict2e, graphicx}
\usepackage{tikz}
\usetikzlibrary{intersections,calc,arrows.meta}

\usepackage{amsmath, amssymb, amsthm}
\usepackage{ascmac}
\usepackage{comment}
\usepackage{empheq}
\usepackage[shortlabels,inline]{enumitem}
\usepackage{fancybox}
\usepackage{fancyhdr}
\usepackage{here}
\usepackage{lastpage}
\usepackage{listings, jvlisting}
\usepackage{fixdif}

\usepackage{stmaryrd}
\usepackage[listings]{tcolorbox}
%\usepackage{ascolorbox}
\usepackage{titlesec}
\usepackage{ulem}
\usepackage{url}
\usepackage{verbatim}
\usepackage{wrapfig}
\usepackage{xcolor}
\usepackage{luatexja-ruby}
\usepackage{varwidth}
\usepackage[version=3]{mhchem}
\usepackage{wrapfig}


\usepackage{physics2}
	\usephysicsmodule{ab}
	\usephysicsmodule{ab.braket}
	\usephysicsmodule{ab.legacy}
	%\usephysicsmodule{braket}
	\usephysicsmodule{diagmat}
	\usephysicsmodule{xmat}
	\usephysicsmodule{nabla.legacy}
	\usephysicsmodule{qtext.legacy}

\usepackage[ISO]{diffcoeff}
\difdef { f, s } { D }
{ op-symbol = \mathrm{D} }


\newcommand{\mctext}[1]{\mbox{\textcircled{\scriptsize{#1}}}}
\newcommand{\ctext}[1]{\textcircled{\scriptsize{#1}}}
\newcommand{\ds}{\displaystyle}
\newcommand{\comb}[2]{{}_{#1}\mathrm{C}_{#2}}
\newcommand{\hs}{\hspace}
\newcommand{\vs}{\vspace}
\newcommand{\emphvs}{\vspace{1em}\notag\\}
\newcommand{\ora}{\overrightarrow}
\newcommand{\ol}{\overline}
\newcommand{\oramr}[1]{\overrightarrow{\mathrm{#1}}}
\newcommand{\tri}{\triangle}
\newcommand{\mr}{\mathrm}
\newcommand{\mb}{\mathbb}
\newcommand{\mrvec}[1]{\overrightarrow{\mathrm{#1}}}
\newcommand{\itvec}{\overrightarrow}
\newcommand{\bs}{\boldsymbol}
\newcommand{\ra}{\rightarrow}
\newcommand{\Ra}{\Rightarrow}
\newcommand{\lra}{\longrightarrow}
\newcommand{\Lra}{\Longrightarrow}
\newcommand{\la}{\leftarrow}
\newcommand{\La}{\Leftarrow}
\newcommand{\lla}{\longleftarrow}
\newcommand{\Lla}{\Longleftarrow}
\newcommand{\lr}{\leftrightarrow}
\newcommand{\llr}{\longleftrightarrow}
\newcommand{\Llr}{\Longleftrightarrow}
\renewcommand{\deg}{{}^\circ}
\newcommand{\phbox}{\fbox{\phantom{1\hspace{2em}}}}
\newcommand{\boxnum}[1]{\fbox{\phantom{\hspace{1em}}({#1})\phantom{\hspace{1em}}}}
\newcommand{\boxkana}[1]{\fbox{\phantom{\hspace{1em}}{#1}\phantom{\hspace{1em}}}}
\newcommand{\boxkm}[2]{\fbox{\, {#1}\phantom{\hspace{0.2em}} \,  {#2}}}
\newcommand{\hzw}{\hspace{1\zw}}

\renewcommand{\baselinestretch}{1.25}
\parindent=1\zw


%%入114

\begin{document}
\noindent
\fbox{NewTH2-2} [大阪大2005]

図1のように,鉛直に立てられている断面積$S$の十分に長いシリンダー内に,なめらかに動く質量$M$のピストンがある.これらは断熱材でできており,外側は真空である.
ピストンは,シリンダーの底面から高さ$d$の位置に,初めは固定されている.ピストンの下のシリンダー内には,単原子分子からなる2種類の理想気体A,Bが均一に混ざって入っており,絶対温度$T$の熱平衡状態になっている.ピストンのすぐ下には,Aの分子だけが抵抗なく通り抜けられるフィルターが,シリンダーに固定されている.フィルターの厚さは無視できるものとする.以下の文中の\phbox{}に適切な数式を書き入れよ.ただし,重力加速度の大きさを$g$とし,重力はピストンのみにはたらくとする.

まず,A,Bそれぞれが壁に及ぼす圧力$p_\mr{A}$,$p_\mr{B}$を,理想気体の分子運動から考えてみよう.圧力は多数の気体分子が容器の壁に衝突することで生じる.この衝突は完全弾性衝突であるとして,質量$m_\mr{A}$のAの分子が速度$\ora{v_\mr{A}}$でピストンに衝突して力を及ぼす場合を考える.鉛直上向きに$z$軸をとる.ピストンが1個の分子から1回の衝突で受ける力積の$z$成分は,$\ora{v_\mr{A}}$の$z$成分$v_{\mr{A}z}$を用いると,\boxkana{ア}と表される.また,この分子が時間$t$の間にピストンに衝突する回数は,\boxkana{イ}と表される.分子の速度は,1つ1つの分子によっていろいろな値をとる.そこで,シリンダー内のAの全分子数を$N_\mr{A}$とすると,Aの全分子がピストンに及ぼす平均の力は$\text{\boxkana{ウ}} \times \braket<v_{\mr{A}z{}}{}^2>$と書ける.ここで,$\braket<v_{\mr{A}z}{}^2>$は$v_{\mr{A}z}$の2乗の平均を表す.気体分子の運動に方向による差はないので,$\braket<v_{\mr{A}z}{}^2>$は,$\ora{v_\mr{A}}$の大きさ$v_\mr{A}$を用いて書き直すことができる.よって,圧力は$p_\mr{A} = \text{\boxkana{エ}}\times \braket<v_\mr{A}{}^2>$となる.同様にして,Bの分子の質量,速さ,シリンダー内の全分子数をそれぞれ$m_\mr{B}$,$v_\mr{B}$,$N_\mr{B}$とすると,$p_\mr{B} = \text{\boxkana{オ}} \times \braket<v_\mr{B}{}^2>$である.$p_\mr{A}$,$p_\mr{B}$のことを,A,Bの分圧と呼ぶ.A,B合わせた気体全体の圧力は分圧の和になっている.ここで,ボルツマン定数を$k$とすると,分子1個あたりの平均の運動エネルギーは$\dfrac{3}{2}kT$なので,$\dfrac{p_\mr{A}}{p_\mr{B}}$は,分子の速さや質量によらず,\boxkana{カ}と書ける.

次に,ピストンの固定を外した.すると,ピストンは上方へ動き,最初の位置から$h$だけ上方でピストンの速さが0になった.その瞬間に,図2のようにピストンを固定した.ピストンの移動に伴ってAがピストンにする仕事は,ピストンの力学的エネルギーの変化に一致する.これより,Aがピストンにした仕事は\boxkana{キ}である.

しばらくすると,A,Bともに温度$T'$の熱平衡状態になった.この状態を熱力学の第1法則を用いて考えよう.温度$T'$はA,B合わせた気体全体の内部エネルギーの変化に着目すると,$T' = \text{\boxkana{ク}} \times Mgh + T$と表される.また,ピストンの固定を外してから温度が$T'$となるまでにBが得た熱量$Q_\mr{B}$は,Bの内部エネルギーの変化に着目し,$T$と$T'$を含んだ式で表すと,\boxkana{ケ}と求まる.このとき,Aが得た熱量$Q_\mr{A}$は,\boxkana{コ}である.

最後に,この熱平衡状態での圧力について考えてみよう.A,Bそれぞれの分圧を$p_\mr{A}{}'$,$p_\mr{B}{}'$とする.\boxkana{カ}で$\dfrac{p_\mr{A}}{p_\mr{B}}$を求めた考え方を,フィルターの下のシリンダー内の分子について適用すると,$p_\mr{A}{}'$と$p_\mr{B}{}'$の圧力の比は,$\dfrac{p_\mr{A}{}'}{p_\mr{B}{}'} = \text{\boxkana{サ}} \times \dfrac{p_\mr{A}}{p_\mr{B}}$と表される.また,この状態ではフィルターの上下に圧力差が生じている.下の圧力は上の圧力に比べて$p_\mr{B} + \text{\boxkana{シ}} \times Q_\mr{B}$だけ大きくなっている.

\end{document}