%%%%%%%%%%%%%%%%%%%%%%%%%%
%%% author : Yamada. T %%%
%%% made for TH series %%%
%%%%%%%%%%%%%%%%%%%%%%%%%%

\documentclass[b5paper,10pt,fleqn] {ltjsarticle}

\usepackage[margin=10truemm]{geometry}

\usepackage{pict2e, graphicx}
\usepackage{tikz}
\usetikzlibrary{intersections,calc,arrows.meta}

\usepackage{amsmath, amssymb, amsthm}
\usepackage{ascmac}
\usepackage{comment}
\usepackage{empheq}
\usepackage[shortlabels,inline]{enumitem}
\usepackage{fancybox}
\usepackage{fancyhdr}
\usepackage{here}
\usepackage{lastpage}
\usepackage{listings, jvlisting}
\usepackage{fixdif}

\usepackage{stmaryrd}
\usepackage[listings]{tcolorbox}
%\usepackage{ascolorbox}
\usepackage{titlesec}
\usepackage{ulem}
\usepackage{url}
\usepackage{verbatim}
\usepackage{wrapfig}
\usepackage{xcolor}
\usepackage{luatexja-ruby}
\usepackage{varwidth}
\usepackage[version=3]{mhchem}
\usepackage{wrapfig}


\usepackage{physics2}
	\usephysicsmodule{ab}
	\usephysicsmodule{ab.braket}
	\usephysicsmodule{ab.legacy}
	%\usephysicsmodule{braket}
	\usephysicsmodule{diagmat}
	\usephysicsmodule{xmat}
	\usephysicsmodule{nabla.legacy}
	\usephysicsmodule{qtext.legacy}

\usepackage[ISO]{diffcoeff}
\difdef { f, s } { D }
{ op-symbol = \mathrm{D} }


\newcommand{\mctext}[1]{\mbox{\textcircled{\scriptsize{#1}}}}
\newcommand{\ctext}[1]{\textcircled{\scriptsize{#1}}}
\newcommand{\ds}{\displaystyle}
\newcommand{\comb}[2]{{}_{#1}\mathrm{C}_{#2}}
\newcommand{\hs}{\hspace}
\newcommand{\vs}{\vspace}
\newcommand{\emphvs}{\vspace{1em}\notag\\}
\newcommand{\ora}{\overrightarrow}
\newcommand{\oramr}[1]{\overrightarrow{\mathrm{#1}}}
\newcommand{\tri}{\triangle}
\newcommand{\mr}{\mathrm}
\newcommand{\mb}{\mathbb}
\newcommand{\mrvec}[1]{\overrightarrow{\mathrm{#1}}}
\newcommand{\itvec}{\overrightarrow}
\newcommand{\bs}{\boldsymbol}
\newcommand{\ra}{\rightarrow}
\newcommand{\Ra}{\Rightarrow}
\newcommand{\lra}{\longrightarrow}
\newcommand{\Lra}{\Longrightarrow}
\newcommand{\la}{\leftarrow}
\newcommand{\La}{\Leftarrow}
\newcommand{\lla}{\longleftarrow}
\newcommand{\Lla}{\Longleftarrow}
\newcommand{\lr}{\leftrightarrow}
\newcommand{\llr}{\longleftrightarrow}
\newcommand{\Llr}{\Longleftrightarrow}
\renewcommand{\deg}{{}^\circ}
\newcommand{\phbox}{\fbox{\phantom{1\hspace{2em}}}}
\newcommand{\boxnum}[1]{\fbox{\phantom{\hspace{1em}}({#1})\phantom{\hspace{1em}}}}
\newcommand{\boxkana}[1]{\fbox{\phantom{\hspace{1em}}{#1}\phantom{\hspace{1em}}}}
\newcommand{\boxkm}[2]{\fbox{\, {#1}\phantom{\hspace{0.2em}} \,  ${#2}$}}
\newcommand{\hzw}{\hspace{1\zw}}

\renewcommand{\baselinestretch}{1.25}
\parindent=1\zw


%TH3-6

\begin{document}
\noindent
\fbox{NewTH1-24} [名古屋大]

質量$m$の探査機を地球上から打ち上げ,火星表面に着陸させることを考える.

\begin{enumerate}[label=\textbf{問\arabic*}]
  \item {\hzw}はじめに,探査機を地球のまわりで地表すれすれの軌道をまわらせるとしよう.このとき必要な速さ$V_1$を,地表における重力加速度の大きさ$g$と地球の半径$R$を用いて表せ.
  \item {\hzw}次に,探査機を加速して,地球の引力を振り切って飛び出させる.探査機がいったん地球の引力を振り切った後は,太陽の引力だけを受けて,地球の公転軌道とまったく同じ軌道を地球と同じ速さでまわる人工衛星になると仮定する.この速さを求めよう.地球の公転軌道は,太陽を中心とした半径$r$の円であると近似する.地球が太陽のまわりを公転する速さ$v_E$を,地球公転軌道の半径$r$と万有引力定数$G$,太陽の質量$M$を用いて表せ.
  \item {\hzw}さらに,探査機をごく短い時間だけ加速して,火星に向かう軌道に乗せるとしよう.探査機が描く軌道は,図1のように,太陽を1つの焦点とし,地球の公転軌道に点Aで接し,火星の公転軌道に点Bで接する楕円軌道としたい.点Bは太陽をはさんで点Aの反対側の点である.なお,火星の公転軌道は,太陽を中心とした半径$Cr$(地球の公転軌道の半径$r$の$C$倍)の円であると近似する.
  \begin{enumerate}[(1)]
    \item {\hzw}ケプラーの第2法則によると,惑星と太陽とを結ぶ線分が一定時間に描く面積は一定である.探査機を楕円軌道を描く惑星とみなしてケプラーの第2法則を用いることによって,点Aで探査機が太陽の周りを公転する速さ$v_\mr{A}$と点Bで探査機が太陽の周りを公転する速さ$v_\mr{B}$との間の関係を求めよ.
    \item {\hzw}探査機が点Aでもつべき運動エネルギーを,火星と地球の公転軌道半径の比$C$と探査機の質量$m$,地球の公転速度$v_E$を用いて表せ.
    \item {\hzw}ケプラーの第3法則によると,各惑星の公転周期$T$と軌道の半長軸$a$の間には,$T^2 = ka^3$($k$は比例定数)が成り立つ.この探査機の軌道の半長軸は線分ABの半分である.探査機が楕円軌道を半周して点Bに達するのは,点Aを出発してから何年後か.$C$を用いて表せ.また,$C$が1.5だとすると,これは何年か,有効数字1桁で答えよ.
  \end{enumerate}
  \item {\hzw}探査機が点Bに達した時に火星も点Bに達し,探査機の速度をうまく制御して火星に衝突させることができたとする.
  {\hzw}探査機は,図2のように,火星の水平な表面に対してある角度をなして速さ$V$で衝突した後,水平から$45\deg$の方向にはね返った.火星の表面はなめらかな面であるとする.はね返った探査機が,火星の表面から25 mの高さの最高点に達した.はね返った直後の速さ$V'$は秒速何mか.有効数字1桁で答えよ.ただし,地球の表面での重力加速度の大きさ$g$は10 m/s${}^2$,火星の質量は地球の質量の0.1倍,火星の半径は地球の半径の0.5倍とする.
\end{enumerate}
\end{document}