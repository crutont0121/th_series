%%%%%%%%%%%%%%%%%%%%%%%%%%
%%% author : Yamada. T %%%
%%% made for TH series %%%
%%%%%%%%%%%%%%%%%%%%%%%%%%

\documentclass[b5paper,10pt,fleqn] {ltjsarticle}

\usepackage[margin=10truemm]{geometry}

\usepackage{pict2e, graphicx}
\usepackage{tikz}
\usetikzlibrary{intersections,calc,arrows.meta}

\usepackage{amsmath, amssymb, amsthm}
\usepackage{ascmac}
\usepackage{comment}
\usepackage{empheq}
\usepackage[shortlabels,inline]{enumitem}
\usepackage{fancybox}
\usepackage{fancyhdr}
\usepackage{here}
\usepackage{lastpage}
\usepackage{listings, jvlisting}
\usepackage{fixdif}

\usepackage{stmaryrd}
\usepackage[listings]{tcolorbox}
%\usepackage{ascolorbox}
\usepackage{titlesec}
\usepackage{ulem}
\usepackage{url}
\usepackage{verbatim}
\usepackage{wrapfig}
\usepackage{xcolor}
\usepackage{luatexja-ruby}
\usepackage{varwidth}
\usepackage[version=3]{mhchem}
\usepackage{wrapfig}


\usepackage{physics2}
	\usephysicsmodule{ab}
	\usephysicsmodule{ab.braket}
	\usephysicsmodule{ab.legacy}
	%\usephysicsmodule{braket}
	\usephysicsmodule{diagmat}
	\usephysicsmodule{xmat}
	\usephysicsmodule{nabla.legacy}
	\usephysicsmodule{qtext.legacy}

\usepackage[ISO]{diffcoeff}
\difdef { f, s } { D }
{ op-symbol = \mathrm{D} }


\newcommand{\mctext}[1]{\mbox{\textcircled{\scriptsize{#1}}}}
\newcommand{\ctext}[1]{\textcircled{\scriptsize{#1}}}
\newcommand{\ds}{\displaystyle}
\newcommand{\comb}[2]{{}_{#1}\mathrm{C}_{#2}}
\newcommand{\hs}{\hspace}
\newcommand{\vs}{\vspace}
\newcommand{\emphvs}{\vspace{1em}\notag\\}
\newcommand{\ora}{\overrightarrow}
\newcommand{\oramr}[1]{\overrightarrow{\mathrm{#1}}}
\newcommand{\tri}{\triangle}
\newcommand{\mr}{\mathrm}
\newcommand{\mb}{\mathbb}
\newcommand{\mrvec}[1]{\overrightarrow{\mathrm{#1}}}
\newcommand{\itvec}{\overrightarrow}
\newcommand{\bs}{\boldsymbol}
\newcommand{\ra}{\rightarrow}
\newcommand{\Ra}{\Rightarrow}
\newcommand{\lra}{\longrightarrow}
\newcommand{\Lra}{\Longrightarrow}
\newcommand{\la}{\leftarrow}
\newcommand{\La}{\Leftarrow}
\newcommand{\lla}{\longleftarrow}
\newcommand{\Lla}{\Longleftarrow}
\newcommand{\lr}{\leftrightarrow}
\newcommand{\llr}{\longleftrightarrow}
\newcommand{\Llr}{\Longleftrightarrow}
\renewcommand{\deg}{{}^\circ}
\newcommand{\phbox}{\fbox{\phantom{1\hspace{2em}}}}
\newcommand{\boxnum}[1]{\fbox{\phantom{\hspace{1em}}({#1})\phantom{\hspace{1em}}}}
\newcommand{\boxkana}[1]{\fbox{\phantom{\hspace{1em}}{#1}\phantom{\hspace{1em}}}}
\newcommand{\boxkm}[2]{\fbox{\, {#1}\phantom{\hspace{0.2em}} \,  ${#2}$}}
\newcommand{\hzw}{\hspace{1\zw}}

\renewcommand{\baselinestretch}{1.25}
\parindent=1\zw

%%入94
\begin{document}
\noindent
\fbox{NewTH1-22} [東京大2005]

図のように,地球の中心Oを通り,地表のある地点Aと地点Bとを結ぶ細長いトンネル内における小球の直線運動を考える.
地球を半径$R$,一様な密度$\rho$の球とみなし,万有引力定数を$G$として,次の問いに答えよ.
なお,地球の中心Oから距離$r$の位置において,小球が地球から受ける力は,中心Oから距離$r$以内にある地球の部分の質量が中心Oに集まったと仮定した場合に,小球が受ける万有引力に等しい.
ただし,地球の自転と公転の影響,トンネルと小球の間の摩擦および空気抵抗は無視するものとし,地球の質量は小球の質量に比べて十分に大きいものとする.

\begin{enumerate}[I]
  \item {\hzw}質量$m$の小球を地点Aから静かにはなしたときの運動を考える.
  \begin{enumerate}[(1)]
    \item {\hzw}小球が地球の中心Oから距離$r\,(r < R)$の位置にあるとき,小球にはたらく力の大きさを求めよ.
    \item {\hzw}小球が運動を開始した後,初めて地点Aに戻ってくるまでの時間$T$を求めよ.
  \end{enumerate}
  \item {\hzw}同じ質量$m$をもつ2つの小球P,Qの運動を考える.時刻0には小球Pを,時刻$t_1$には小球Qを同一の地点Aで静かにはなしたところ,2つの小球はOBの中点Cで衝突した.
  ここで2つの小球の間の反発係数を0とし,衝突後,2つの小球は一体となって運動するものとする.ただし,$t_1$は(2)で求めた時間$T$より小さいものとする.
  \begin{enumerate}[(1)]
    \item {\hzw}$t_1$を$T$を用いて表せ.
    \item {\hzw}2つの小球P,Qが衝突してから初めて中心Oを通過するまでの時間を$T$を用いて表せ.
  \end{enumerate}
  \item {\hzw}IIと同様に,時刻0に小球Pを,時刻$t_1$に小球Qを同一の地点Aで静かにはなした.ただし,2つの小球の間の反発係数を$e\, (0 < e < 1)$とする.
  \begin{enumerate}[(1)]
    \item {\hzw}2つの小球が最初に衝突した後,小球Pは地点Bに向かって運動し,地球の中心Oから距離$d$の点Dにおいて,中心Oに向かって折り返した.このときの$d$の値を反発係数$e$および地球の半径$R$を用いて表せ.
    \item {\hzw}小球Pと小球Qが2回目に衝突する位置を求めよ.
    \item {\hzw}その後2つの小球は衝突を繰り返した.十分に時間が経過した後,どのような運動になるか答えよ.
  \end{enumerate}
\end{enumerate}

\end{document}