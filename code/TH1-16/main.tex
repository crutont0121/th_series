%%%%%%%%%%%%%%%%%%%%%%%%%%
%%% author : Yamada. T %%%
%%% made for TH series %%%
%%%%%%%%%%%%%%%%%%%%%%%%%%

\documentclass[b5paper,10pt,fleqn] {ltjsarticle}

\usepackage[margin=10truemm]{geometry}

\usepackage{pict2e, graphicx}
\usepackage{tikz}
\usetikzlibrary{intersections,calc,arrows.meta}

\usepackage{amsmath, amssymb, amsthm}
\usepackage{ascmac}
\usepackage{comment}
\usepackage{empheq}
\usepackage[shortlabels,inline]{enumitem}
\usepackage{fancybox}
\usepackage{fancyhdr}
\usepackage{here}
\usepackage{lastpage}
\usepackage{listings, jvlisting}
\usepackage{fixdif}

\usepackage{stmaryrd}
\usepackage[listings]{tcolorbox}
%\usepackage{ascolorbox}
\usepackage{titlesec}
\usepackage{ulem}
\usepackage{url}
\usepackage{verbatim}
\usepackage{wrapfig}
\usepackage{xcolor}
\usepackage{luatexja-ruby}
\usepackage{varwidth}
\usepackage[version=3]{mhchem}
\usepackage{wrapfig}


\usepackage{physics2}
	\usephysicsmodule{ab}
	\usephysicsmodule{ab.braket}
	\usephysicsmodule{ab.legacy}
	%\usephysicsmodule{braket}
	\usephysicsmodule{diagmat}
	\usephysicsmodule{xmat}
	\usephysicsmodule{nabla.legacy}
	\usephysicsmodule{qtext.legacy}

\usepackage[ISO]{diffcoeff}
\difdef { f, s } { D }
{ op-symbol = \mathrm{D} }


\newcommand{\mctext}[1]{\mbox{\textcircled{\scriptsize{#1}}}}
\newcommand{\ctext}[1]{\textcircled{\scriptsize{#1}}}
\newcommand{\ds}{\displaystyle}
\newcommand{\comb}[2]{{}_{#1}\mathrm{C}_{#2}}
\newcommand{\hs}{\hspace}
\newcommand{\vs}{\vspace}
\newcommand{\emphvs}{\vspace{1em}\notag\\}
\newcommand{\ora}{\overrightarrow}
\newcommand{\ol}{\overline}
\newcommand{\oramr}[1]{\overrightarrow{\mathrm{#1}}}
\newcommand{\tri}{\triangle}
\newcommand{\mr}{\mathrm}
\newcommand{\mb}{\mathbb}
\newcommand{\mrvec}[1]{\overrightarrow{\mathrm{#1}}}
\newcommand{\itvec}{\overrightarrow}
\newcommand{\bs}{\boldsymbol}
\newcommand{\ra}{\rightarrow}
\newcommand{\Ra}{\Rightarrow}
\newcommand{\lra}{\longrightarrow}
\newcommand{\Lra}{\Longrightarrow}
\newcommand{\la}{\leftarrow}
\newcommand{\La}{\Leftarrow}
\newcommand{\lla}{\longleftarrow}
\newcommand{\Lla}{\Longleftarrow}
\newcommand{\lr}{\leftrightarrow}
\newcommand{\llr}{\longleftrightarrow}
\newcommand{\Llr}{\Longleftrightarrow}
\renewcommand{\deg}{{}^\circ}
\newcommand{\phbox}{\fbox{\phantom{1\hspace{2em}}}}
\newcommand{\boxnum}[1]{\fbox{\phantom{\hspace{1em}}({#1})\phantom{\hspace{1em}}}}
\newcommand{\boxkana}[1]{\fbox{\phantom{\hspace{1em}}{#1}\phantom{\hspace{1em}}}}
\newcommand{\boxkm}[2]{\fbox{\, {#1}\phantom{\hspace{0.2em}} \,  {#2}}}
\newcommand{\hzw}{\hspace{1\zw}}

\renewcommand{\baselinestretch}{1.25}
\parindent=1\zw

%東京大2002-1
\begin{document}
\noindent
\fbox{NewTH1-16} [東京大2002]


長さ$L$の不透明な細いパイプの中に,質量$m$の小球1と質量$2m$の小球2が埋め込まれている.パイプは直線状で曲がらず,その口径,及び小球以外の部分の質量は無視できるほど小さい.また小球は質点とみなしてよいとし,重力加速度を$g$とする.これらの小球の位置を調べるために次の二つの実験を行った.

\begin{enumerate}[I]
  \item {\hzw}まず,図1--1に示したように,パイプの両端A,Bを支点a,bで支え,両方の支点を近づけるような力をゆっくりとかけていったところ,まずbがCの位置まで滑って止まり,その直後に今度はaが滑り出してDの位置で止まった.パイプと支点の間の静止摩擦係数,及び動摩擦係数をそれぞれ$\mu$,$\mu'$(ただし,$\mu > \mu'$)と記すことにして,以下の問に答えよ.
  \begin{enumerate}[(1)]
    \item {\hzw}bがCで止まる直前に支点a,bにかかっているパイプに垂直な方向の力をそれぞれ$N_\mr{a}$,$N_\mr{b}$とする.このときのパイプに沿った方向の力のつり合いを表す式を書け.
    \item {\hzw}ACの長さを測定したところ$d_1$であった.パイプの重心が左端Aから測って$l$の位置にあるとするとき,重心の周りの力のモーメントのつり合いを考えることにより.$d_1$を$l$,$\mu$,$\mu'$を用いて表せ.
    \item {\hzw}CDの長さを測定したところ$d_2$であった.摩擦係数の比$\dfrac{\mu'}{\mu}$を$d_1$,$d_2$で表せ.
    \item {\hzw}上記の測定から重心の位置$l$を求めることができる.$l$を$d_1$,$d_2$で表せ.
    \item {\hzw}さらに両方の支点を近づけるプロセスを続けると,どのような現象が起こり,最終的にどのような状態に行き着くか.理由も含めて簡潔に述べよ.
  \end{enumerate}
  \item {\hzw}次に,パイプの端Aに小さな穴を開け,図1--2のようにそこを支点として鉛直に立てた状態から静かにはなし,パイプを回転させた.パイプが$180\deg$回転したときの端Bの速度の大きさを測ったところ,$v$であった.端Aから測った小球1,2の位置をそれぞれ$l_1$,$l_2$として以下の問に答えよ.(支点での摩擦および空気抵抗は無視できるものとする.)
  \begin{enumerate}[(1)]
    \item {\hzw}$v$を$l_1$,$l_2$,$g$,$L$を用いて表せ.
    \item {\hzw}$v$を実験1で得られたれた重心の位置$l$の値を用いて表したところ,
    \begin{gather*}
      v = L \sqrt{\frac{8g}{3l}}
    \end{gather*}
    であった.小球の位置$l_1$,$l_2$を$l$で表せ.ただし$l_1 \neq 0$,$l_2 \neq 0$とする.
  \end{enumerate}
\end{enumerate}
\end{document}