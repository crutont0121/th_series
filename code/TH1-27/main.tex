%%%%%%%%%%%%%%%%%%%%%%%%%%
%%% author : Yamada. T %%%
%%% made for TH series %%%
%%%%%%%%%%%%%%%%%%%%%%%%%%

\documentclass[b5paper,10pt,fleqn] {ltjsarticle}

\usepackage[margin=10truemm]{geometry}

\usepackage{pict2e, graphicx}
\usepackage{tikz}
\usetikzlibrary{intersections,calc,arrows.meta}

\usepackage{amsmath, amssymb, amsthm}
\usepackage{ascmac}
\usepackage{comment}
\usepackage{empheq}
\usepackage[shortlabels,inline]{enumitem}
\usepackage{fancybox}
\usepackage{fancyhdr}
\usepackage{here}
\usepackage{lastpage}
\usepackage{listings, jvlisting}
\usepackage{fixdif}

\usepackage{stmaryrd}
\usepackage[listings]{tcolorbox}
%\usepackage{ascolorbox}
\usepackage{titlesec}
\usepackage{ulem}
\usepackage{url}
\usepackage{verbatim}
\usepackage{wrapfig}
\usepackage{xcolor}
\usepackage{luatexja-ruby}
\usepackage{varwidth}
\usepackage[version=3]{mhchem}
\usepackage{wrapfig}


\usepackage{physics2}
	\usephysicsmodule{ab}
	\usephysicsmodule{ab.braket}
	\usephysicsmodule{ab.legacy}
	%\usephysicsmodule{braket}
	\usephysicsmodule{diagmat}
	\usephysicsmodule{xmat}
	\usephysicsmodule{nabla.legacy}
	\usephysicsmodule{qtext.legacy}

\usepackage[ISO]{diffcoeff}
\difdef { f, s } { D }
{ op-symbol = \mathrm{D} }


\newcommand{\mctext}[1]{\mbox{\textcircled{\scriptsize{#1}}}}
\newcommand{\ctext}[1]{\textcircled{\scriptsize{#1}}}
\newcommand{\ds}{\displaystyle}
\newcommand{\comb}[2]{{}_{#1}\mathrm{C}_{#2}}
\newcommand{\hs}{\hspace}
\newcommand{\vs}{\vspace}
\newcommand{\emphvs}{\vspace{1em}\notag\\}
\newcommand{\ora}{\overrightarrow}
\newcommand{\ol}{\overline}
\newcommand{\oramr}[1]{\overrightarrow{\mathrm{#1}}}
\newcommand{\tri}{\triangle}
\newcommand{\mr}{\mathrm}
\newcommand{\mb}{\mathbb}
\newcommand{\mrvec}[1]{\overrightarrow{\mathrm{#1}}}
\newcommand{\itvec}{\overrightarrow}
\newcommand{\bs}{\boldsymbol}
\newcommand{\ra}{\rightarrow}
\newcommand{\Ra}{\Rightarrow}
\newcommand{\lra}{\longrightarrow}
\newcommand{\Lra}{\Longrightarrow}
\newcommand{\la}{\leftarrow}
\newcommand{\La}{\Leftarrow}
\newcommand{\lla}{\longleftarrow}
\newcommand{\Lla}{\Longleftarrow}
\newcommand{\lr}{\leftrightarrow}
\newcommand{\llr}{\longleftrightarrow}
\newcommand{\Llr}{\Longleftrightarrow}
\renewcommand{\deg}{{}^\circ}
\newcommand{\phbox}{\fbox{\phantom{1\hspace{2em}}}}
\newcommand{\boxnum}[1]{\fbox{\phantom{\hspace{1em}}({#1})\phantom{\hspace{1em}}}}
\newcommand{\boxkana}[1]{\fbox{\phantom{\hspace{1em}}{#1}\phantom{\hspace{1em}}}}
\newcommand{\boxkm}[2]{\fbox{\, {#1}\phantom{\hspace{0.2em}} \,  {#2}}}
\newcommand{\hzw}{\hspace{1\zw}}

\renewcommand{\baselinestretch}{1.25}
\parindent=1\zw


%%TH3-8

\begin{document}
\noindent
\fbox{NewTH1-27} [京都大]

次のI〜IVの文章を読んで,空欄には適した式または数値を,{{\hzw\hzw}}からは適切なものを1つ選びその番号を,それぞれ記せ.数値の場合は単位も明記すること.なお,重力加速度の大きさを$g$とし,浮力は無視して良い.

\begin{enumerate}[I]
  \item {\hzw}質量$m$の物体が重力と抵抗力を受けて鉛直下向きに速度$v$で落下している.抵抗力の大きさは物体の速さに比例すると仮定し,比例定数を$k$とする.また,速度,加速度は鉛直下向きを正にとる.この物体の運動方程式は, 微小時間$\varDelta t$での速度変化を$\varDelta v$とすると
  \begin{gather*}
    m\frac{\varDelta v}{\varDelta t} = mg - kv
  \end{gather*}
  で与えられる.この状況では,落下を開始して一定時間の後には,物体の運動は,近似的に等速度運動になる.このときの速度を終端速度という.終端速度$v_\mr{f}$は重力と抵抗力がつり合う条件で決まり,$v_\mr{f} = \text{\boxnum{1}}$で与えられる.また,終端速度を用いると運動方程式は
  \begin{gather*}
    m\frac{\varDelta v}{\varDelta t} = k(v_\mr{f} - v)\qq{}\cdots \text{(i)}
  \end{gather*}
  と表せる.時間とともに速度$v$がどのように終端速度に近づくか議論しよう.そのため,$v = v_\mr{f} + \ol{v}$として終端速度からのずれ$\ol{v}$を導入すると,式(i)より,
  \begin{gather*}
    \frac{\varDelta \ol{v}}{\varDelta t} = -\frac{\ol{v}}{\text{\boxnum{2}}}
  \end{gather*}
  が導かれる.なお,$\varDelta \ol{v}$は微小時間$\varDelta t$での$\ol{v}$の変化である.ここで,$\tau_1 = \text{\boxnum{2}}$は緩和時間とよばれ,速度が終端速度$v_\mr{f}$に近づく時間の目安である.この場合,緩和時間$\tau_1$と終端速度$v_\mr{f}$との間には$v_\mr{f} = \text{\boxnum{3}}\times \tau_1$という関係がある.

  {\hzw}ここで,2種類の初期条件を考える.一方は初速度0,他方は初速度が終端速度の2倍である.これらの条件における速度の変化を正しく表しているグラフは図1の{(4):①,②,③,④}である.ただし,点線は終端速度を表している.

  %%% fig 1

  \item {\hzw}次に,抵抗力の大きさが物体の速さの2乗に比例する場合を考えよう.鉛直下向きの速度を$v$とすると,物体の運動方程式は
  \begin{gather*}
    m\frac{\varDelta v}{\varDelta t} = mg - cv^2 \qq{} \cdots \text{(ii)}
  \end{gather*}
  で与えられる.定数を抵抗係数とよぶことにする.このとき,終端速度$v_\mr{t}$は$m$,$g$,$c$を用いて$v_\mr{t} = \text{\boxnum{5}}$で与えられる.Iと同様に,時間とともに速度$v$がどのように終端速度に近づくか議論しよう.そのため,$v = v_\mr{t} + \ol{v}$と終端速度からのずれ$\ol{v}$を導入する.速度が終端速度に近い,すなわち$\ab|\ol{v}|$が$v_\mr{t}$より十分小さい($\ab|\ol{v}| \ll v_\mr{t}$)として,$\ol{v}$の1次までで近似すると,終端速度からのずれ$\ol{v}$の時間変化は
  \begin{gather*}
    \frac{\varDelta \ol{v}}{\varDelta t} = -\frac{\ol{v}}{\tau_2}
  \end{gather*}
  と表すことができる.ここで,$\tau_2$は緩和時間とよばれ,物体の速度が終端速度$v_\mr{t}$に近づく時間の目安であり,$m$,$g$,$c$を用いて$\tau_2 = \text{\boxnum{6}}$で与えられる.

  \item {\hzw}水中で物体を静かに落下させ,落下を始めてからの時間と落下距離の関係を計測した.この実験結果について考えよう.なお,重力加速度の大きさ$g$は9.8 m/s${}^2$とする.

  {\hzw}この実験では,一方は質量$m_1 = 1.0 \, \text{kg}$の物体,他方は質量$m_2 = 2.0 \, \text{kg}$の物体と,形状は同じで質量だけ異なる2種類の物体を落下させた.それぞれを実験1,実験2とよぶことにする.2つの実験の結果を表1に示すとともに,物体の時間と落下距離の関係をグラフにすると図2のようになる.
  \begin{table}[H]
    \centering
    \caption{}
    \begin{tabular}[H]{|c|r|r|r|r|c|r|r|r|r|}
      \hline
      \multicolumn{5}{|c|}{$m_1=1.0\,\text{kg}$の物体の結果(実験1)} & \multicolumn{5}{c|}{$m_2 = 2.0 \, \text{kg}$の物体の結果(実験2)}\\
      \hline
      時間〔s〕& 3.0 & 4.0 & 5.0 & 6.0 & 時間〔s〕& 3.0 & 4.0 & 5.0 & 6.0 \\
      \hline
      落下距離〔m〕 & 15.0 & 20.8 & 26.6 & 32.4 & 落下距離〔m〕 & 19.8 & 28.0 & 36.2 & 44.4\\
      \hline
    \end{tabular}
  \end{table}

  %% fig

  {\hzw}質量$m_1 = 1.0 \, \text{kg}$と質量$m_2 = 2.0 \, \text{kg}$の物体の終端速度を$v_1$,$v_2$とする.実験結果より,終端速度の大きさは有効数字2桁で,$v_1 = \text{\boxnum{7}}$,$v_2 = \text{\boxnum{8}}$である.
  
  \vspace{1em}
  \begin{description}[labelsep=1\zw]
    \item[問] {\hzw}IIIの2つの実験結果より,抵抗力の大きさは速さの2乗に比例していると考えられる.その理由を示せ.ただし,抵抗力に関する定数$k$,$c$はそれぞれ物体の形状で決まり,質量に依存しないと考えて良い.
  \end{description}
  \vspace{1em}

  {\hzw}また,実験1,すなわち質量$m_1 = 1.0 \,\text{kg}$の物体を落下させた場合について,実験データから得られた終端速度をもとに緩和時間$\tau_2$の数値を有効数字1桁で計算すると$\tau_2 = \text{\boxnum{9}}$となり,速やかに終端速度に達していることが理解できる.抵抗力の大きさは速さの2乗に比例するとして,物体を静かに落下させてから時間3.0 sまでの速度の変化を実験1,2の両方について正しく描いているのは図3の{(10):①,②,③,④,⑤,⑥}である.
  ただし,2本の点線は実験1,2それぞれの終端速度を表している.

  % fig 3

  \item {\hzw}速さの2乗に比例する抵抗力について簡単な力学モデルを用いてさらに考察する.図4のように,断面積$S$,質量$m$の円柱形の物体が水中を運動している.水から受ける効果だけを考えたいので,物体は水平方向に運動しているとする.水の密度は$\rho$とする.速度,加速度は右向きを正にとり,時刻$t$での物体の速度は$v$とする.ここで,この物体が時刻$t$から微小時間$\varDelta t$の間,物体の前面がこの微小時間に通過する領域を占めていた微小質量$\varDelta m = \rho \times \text{\boxnum{11}} \times \varDelta t$の静止した水のかたまりと衝突すると考える.その結果,時刻$t + \varDelta t$には水のかたまりは物体と一体となって速度$v + \varDelta v$で運動することになる.物体と水のかたまりを合わせた全運動量が保存されるので,微小時間$\varDelta t$の間に生じる微小な速度変化$\varDelta v$より
  \begin{gather*}
    m\frac{\varDelta v}{\varDelta t} = \text{\boxnum{12}} \times v^2
  \end{gather*}
  のように,水のかたまりとの衝突により物体に作用する力を導くことができる.ただし,微小量$\varDelta t$,$\varDelta v$の1次までを残し,2次は無視すること.

\end{enumerate}
\end{document}