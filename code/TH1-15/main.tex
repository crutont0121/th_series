%%%%%%%%%%%%%%%%%%%%%%%%%%
%%% author : Yamada. T %%%
%%% made for TH series %%%
%%%%%%%%%%%%%%%%%%%%%%%%%%

\documentclass[b5paper,10pt,fleqn] {ltjsarticle}

\usepackage[margin=10truemm]{geometry}

\usepackage{pict2e, graphicx}
\usepackage{tikz}
\usetikzlibrary{intersections,calc,arrows.meta}

\usepackage{amsmath, amssymb, amsthm}
\usepackage{ascmac}
\usepackage{comment}
\usepackage{empheq}
\usepackage[shortlabels,inline]{enumitem}
\usepackage{fancybox}
\usepackage{fancyhdr}
\usepackage{here}
\usepackage{lastpage}
\usepackage{listings, jvlisting}
\usepackage{fixdif}

\usepackage{stmaryrd}
\usepackage[listings]{tcolorbox}
%\usepackage{ascolorbox}
\usepackage{titlesec}
\usepackage{ulem}
\usepackage{url}
\usepackage{verbatim}
\usepackage{wrapfig}
\usepackage{xcolor}
\usepackage{luatexja-ruby}
\usepackage{varwidth}
\usepackage[version=3]{mhchem}
\usepackage{wrapfig}


\usepackage{physics2}
	\usephysicsmodule{ab}
	\usephysicsmodule{ab.braket}
	\usephysicsmodule{ab.legacy}
	%\usephysicsmodule{braket}
	\usephysicsmodule{diagmat}
	\usephysicsmodule{xmat}
	\usephysicsmodule{nabla.legacy}
	\usephysicsmodule{qtext.legacy}

\usepackage[ISO]{diffcoeff}
\difdef { f, s } { D }
{ op-symbol = \mathrm{D} }


\newcommand{\mctext}[1]{\mbox{\textcircled{\scriptsize{#1}}}}
\newcommand{\ctext}[1]{\textcircled{\scriptsize{#1}}}
\newcommand{\ds}{\displaystyle}
\newcommand{\comb}[2]{{}_{#1}\mathrm{C}_{#2}}
\newcommand{\hs}{\hspace}
\newcommand{\vs}{\vspace}
\newcommand{\emphvs}{\vspace{1em}\notag\\}
\newcommand{\ora}{\overrightarrow}
\newcommand{\ol}{\overline}
\newcommand{\oramr}[1]{\overrightarrow{\mathrm{#1}}}
\newcommand{\tri}{\triangle}
\newcommand{\mr}{\mathrm}
\newcommand{\mb}{\mathbb}
\newcommand{\mrvec}[1]{\overrightarrow{\mathrm{#1}}}
\newcommand{\itvec}{\overrightarrow}
\newcommand{\bs}{\boldsymbol}
\newcommand{\ra}{\rightarrow}
\newcommand{\Ra}{\Rightarrow}
\newcommand{\lra}{\longrightarrow}
\newcommand{\Lra}{\Longrightarrow}
\newcommand{\la}{\leftarrow}
\newcommand{\La}{\Leftarrow}
\newcommand{\lla}{\longleftarrow}
\newcommand{\Lla}{\Longleftarrow}
\newcommand{\lr}{\leftrightarrow}
\newcommand{\llr}{\longleftrightarrow}
\newcommand{\Llr}{\Longleftrightarrow}
\renewcommand{\deg}{{}^\circ}
\newcommand{\phbox}{\fbox{\phantom{1\hspace{2em}}}}
\newcommand{\boxnum}[1]{\fbox{\phantom{\hspace{1em}}({#1})\phantom{\hspace{1em}}}}
\newcommand{\boxkana}[1]{\fbox{\phantom{\hspace{1em}}{#1}\phantom{\hspace{1em}}}}
\newcommand{\boxkm}[2]{\fbox{\, {#1}\phantom{\hspace{0.2em}} \,  {#2}}}
\newcommand{\hzw}{\hspace{1\zw}}

\renewcommand{\baselinestretch}{1.25}
\parindent=1\zw


\begin{document}
\noindent
\fbox{NewTH1-15} [東京大2006]

太陽系以外で,恒星の周りを公転する惑星が初めて発見されたのは1995年である.
以来,すでに150個以上の太陽系外惑星が発見されている.
この太陽系外惑星の検出原理は,質量$M$の恒星と質量$m$の惑星($M > m$)が,互いの万有引力だけによってそれぞれ運動している場合を考えれば理解できる.
この場合,惑星は一般には\ruby{楕}{だ}円軌道上を運動することが知られている.
しかしここでは図1に示すように,惑星がある定点Cを中心とした半径$a$の円周上を等速円運動しているとする(ただし,図1には恒星を図示していないことに注意).万有引力定数を$G$とし,恒星および惑星の大きさは無視する.

\begin{enumerate}[label=\textbf{問\arabic*}]
  \item {\hzw}図1のように,惑星が反時計回りに公転しているものとする.惑星に働く向心力は恒星による万有引力であることを考えて,以下の問に答えよ.
  \begin{enumerate}[(1)]
    \item {\hzw}恒星,惑星,点Cの互いの位置関係を理由とともに述べよ.
    \item {\hzw}恒星と点Cとの距離,惑星の速さ$v$,恒星の速さ$V$を求めよ.
    \item {\hzw}惑星の公転軌道面上において,$a$に比べて十分遠方にあり,点Cに対して静止している観測者を考える.図1のように惑星が角度$\theta \, [\mr{rad}]$の位置にあるとき,惑星の速度の視線方向成分$v_r$を,$v$と$\theta$を用いて表せ.ただし,観測者に対して遠ざかる向きを$v_r$の正の向きに選ぶものとする.
    \item {\hzw}時刻$t = 0$において,惑星が$\theta = 0$の位置にあったとする.また,惑星の公転周期を$T$,恒星の速度の視線方向性文を$V_r$とする.$v_r$と$V_r$を$t$の関数として,その概形を$-\dfrac{T}{2} \leqq t \leqq \dfrac{T}{2}$の範囲でグラフに描け.ただし,観測者に対して遠ざかる向きを$v_r$と$V_r$の正の向きに選ぶものとする.
    \end{enumerate}
    \item {\hzw}惑星からの光は弱すぎて観測することは困難である.しかし,恒星からの光を観測することによって,惑星の存在を知ることができる.この間接的な惑星検出方法では,運動する恒星が発する光の波長は,音源が動いた場合の音の波長と同様に,ドップラー効果によって変化することを利用する.ここでは,恒星が静止している場合には波長$\lambda_0$の光を発するものとして,以下の問に答えよ.
    \begin{enumerate}[(1)]
      \item {\hzw}惑星が角度$\theta$の位置にあるときに恒星が発する光を観測者が測定したところ,波長は$\lambda$であった.光速度を$c$として,波長の変化量$\varDelta \lambda = \lambda - \lambda_0$を$\theta$の関数として求めよ.
      \item {\hzw}問2(1)で求めた$\varDelta \lambda$は時間変動する.$0 \leqq \theta < 2\pi$の範囲で$\ab|\dfrac{\varDelta \lambda}{\lambda_0}|$の最大値が$10^{-7}$以上であれば,現在の観測技術で$\varDelta \lambda$の時間変動を検出することができる.このことから,惑星の存在を知ることが可能であるために$a$が満たすべき条件式を求めよ.
      \item {\hzw}問2(2)において,恒星が太陽質量$M_\mr{s} = 2 \times 10^{30} \, \mr{kg}$,惑星が木星程度の質量$10^{-3}M_\mr{s}$を持つものとする.この惑星が検出可能であるために公転周期$T$が満たすべき条件を,有効数字1桁で表せ.ただし,$G = 7 \times 10^{-11}\,\mr{N}\cdot\mr{m}^2/\mr{kg}^2$,$c = 3 \times 10^{8} \, \mr{m/s}$とする.
    \end{enumerate}

\end{enumerate}
\end{document}