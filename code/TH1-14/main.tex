%%%%%%%%%%%%%%%%%%%%%%%%%%
%%% author : Yamada. T %%%
%%% made for TH series %%%
%%%%%%%%%%%%%%%%%%%%%%%%%%

\documentclass[b5paper,10pt,fleqn] {ltjsarticle}

\usepackage[margin=10truemm]{geometry}

\usepackage{pict2e, graphicx}
\usepackage{tikz}
\usetikzlibrary{intersections,calc,arrows.meta}

\usepackage{amsmath, amssymb, amsthm}
\usepackage{ascmac}
\usepackage{comment}
\usepackage{empheq}
\usepackage[shortlabels,inline]{enumitem}
\usepackage{fancybox}
\usepackage{fancyhdr}
\usepackage{here}
\usepackage{lastpage}
\usepackage{listings, jvlisting}
\usepackage{fixdif}

\usepackage{stmaryrd}
\usepackage[listings]{tcolorbox}
%\usepackage{ascolorbox}
\usepackage{titlesec}
\usepackage{ulem}
\usepackage{url}
\usepackage{verbatim}
\usepackage{wrapfig}
\usepackage{xcolor}
\usepackage{luatexja-ruby}
\usepackage{varwidth}
\usepackage[version=3]{mhchem}
\usepackage{wrapfig}


\usepackage{physics2}
	\usephysicsmodule{ab}
	\usephysicsmodule{ab.braket}
	\usephysicsmodule{ab.legacy}
	%\usephysicsmodule{braket}
	\usephysicsmodule{diagmat}
	\usephysicsmodule{xmat}
	\usephysicsmodule{nabla.legacy}
	\usephysicsmodule{qtext.legacy}

\usepackage[ISO]{diffcoeff}
\difdef { f, s } { D }
{ op-symbol = \mathrm{D} }


\newcommand{\mctext}[1]{\mbox{\textcircled{\scriptsize{#1}}}}
\newcommand{\ctext}[1]{\textcircled{\scriptsize{#1}}}
\newcommand{\ds}{\displaystyle}
\newcommand{\comb}[2]{{}_{#1}\mathrm{C}_{#2}}
\newcommand{\hs}{\hspace}
\newcommand{\vs}{\vspace}
\newcommand{\emphvs}{\vspace{1em}\notag\\}
\newcommand{\ora}{\overrightarrow}
\newcommand{\oramr}[1]{\overrightarrow{\mathrm{#1}}}
\newcommand{\tri}{\triangle}
\newcommand{\mr}{\mathrm}
\newcommand{\mb}{\mathbb}
\newcommand{\mrvec}[1]{\overrightarrow{\mathrm{#1}}}
\newcommand{\itvec}{\overrightarrow}
\newcommand{\bs}{\boldsymbol}
\newcommand{\ra}{\rightarrow}
\newcommand{\Ra}{\Rightarrow}
\newcommand{\lra}{\longrightarrow}
\newcommand{\Lra}{\Longrightarrow}
\newcommand{\la}{\leftarrow}
\newcommand{\La}{\Leftarrow}
\newcommand{\lla}{\longleftarrow}
\newcommand{\Lla}{\Longleftarrow}
\newcommand{\lr}{\leftrightarrow}
\newcommand{\llr}{\longleftrightarrow}
\newcommand{\Llr}{\Longleftrightarrow}
\renewcommand{\deg}{{}^\circ}
\newcommand{\phbox}{\fbox{\phantom{1\hspace{2em}}}}
\newcommand{\boxnum}[1]{\fbox{\phantom{\hspace{1em}}({#1})\phantom{\hspace{1em}}}}
\newcommand{\boxkana}[1]{\fbox{\phantom{\hspace{1em}}{#1}\phantom{\hspace{1em}}}}
\newcommand{\boxkm}[2]{\fbox{\, {#1}\phantom{\hspace{0.2em}} \,  ${#2}$}}
\newcommand{\hzw}{\hspace{1\zw}}

\renewcommand{\baselinestretch}{1.25}
\parindent=1\zw


\begin{document}
\noindent
\fbox{NewTH1-14} [大阪大2009]


天井からばね定数$k$のばねによって質量$m$の小球がつり下げられている.
鉛直上向きを$x$軸の正の向きとし,小球にはたらく重力と弾性力がつり合う位置を原点($x = 0$),床の位置を$x = x_1$とする.図のように,小球を$x = h \ (h > 0)$まで鉛直上向きに持ち上げ,時刻$t = 0$に静かに手を離したときの小球の運動について考える.以下の問いに答えよ.
重力加速度の大きさを$g$とし,空気の抵抗およびばねの重さは無視できるものとする.
\begin{enumerate}[label=\textbf{問\arabic*}]
  \item {\hzw}$x_1 < -h$のとき,小球は床と衝突せず単振動をする.
  \begin{enumerate}[(1)]
    \item {\hzw}単振動の周期$T$を$m, k$を用いて表せ.
  \end{enumerate}
  \item {\hzw}次に,床の位置$x_1$が$-h < x_1 < 0$であり,小球が床と弾性衝突する場合について考える.
  \begin{enumerate}[(1), resume]
    \item {\hzw}小球の位置$x$を時刻$t$の関数として図示せよ.なお,右のグラフに示してある曲線は,床が存在しない場合の小球の運動を表している.
    \item {\hzw}以下の文の空欄にふさわしい式を記せ.ただし,各式は各欄に記載した文字のうち必要なものを用いて表せ.
    
    {\hzw}小球は一定の時間間隔$\varDelta t$で床と衝突を繰り返す.小球が最初に床と衝突する時刻を$t = t_1$とすると,時間間隔は\boxkm{ア}{t_1, T}となる.ここで,小球の運動を図のように半径$h$,周期$T$の時計回りの等速円運動と対応させて考えてみる.時刻$t = 0$における円運動の回転角を$\theta = 0$とすると,小球が初めて床と衝突する時刻$t = t_1$での回転角$\theta_1 \text{〔rad〕}$は$\theta_1 = \text{\boxkm{イ}{t_1, T}} = \text{\boxkm{ウ}{T, \varDelta t}}$である.したがって,$\varDelta t$は$\cos\text{\boxkana{ウ}} = \text{\boxkm{エ}{h, x_1}}$の関係を満たすことがわかる.
  \end{enumerate}
  \item {\hzw}次に床の位置$x_1$は問2と同じであるが,小球と床との衝突が非弾性衝突である場合について考える.ただし,小球と床の反発係数を$e\ (0 < e <1)$とする.
  \begin{enumerate}[(1), resume]
    \item 以下の文章中の空欄アとイにふさわしい式を記せ.各式は,各欄に記載した文字のうち必要なものを用いて表せ.また,ウは正しいものを選べ.
    
    {\hzw}時刻$t = t_1$に,小球が床と初めて衝突する直前の小球の速さ$v_0$は$v_0 = \text{\boxkm{ア}{m, k, h, x_1}}$である.

    {\hzw}床と衝突した後,小球は位置$h_1 = \text{\boxkm{イ}{v_0, e, m, k, x_1}}$まで到達し,再び落下して床と衝突する.問2で考えた,小球と床が弾性衝突する場合の衝突の時間間隔$\varDelta t$と比較すると,小球が初めて床と衝突してから次に床と衝突するまでの時間間隔は{ウ{\hzw}①長くなる{\hzw}②変化しない{\hzw}③短くなる}.

    \item{\hzw}小球は床と衝突を繰り返す.$n$回目の衝突直後の小球の速度$v_n$と,その後$(n+1)$回目に衝突するまでに到達する最高点の位置$h_n$を$v_0, e, m, k, x_1, n$のうち必要なものを用いて表せ.
    \item {\hzw}小球が十分に多数回床と衝突を繰り返した後,小球はどのような運動をしているか述べよ.
    \item {\hzw}$t = 0$から(6)の運動になるまでに失った力学的エネルギーの大きさを求めよ.
  \end{enumerate}
\end{enumerate}
\end{document}