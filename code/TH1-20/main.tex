%%%%%%%%%%%%%%%%%%%%%%%%%%
%%% author : Yamada. T %%%
%%% made for TH series %%%
%%%%%%%%%%%%%%%%%%%%%%%%%%

\documentclass[b5paper,10pt,fleqn] {ltjsarticle}

\usepackage[margin=10truemm]{geometry}

\usepackage{pict2e, graphicx}
\usepackage{tikz}
\usetikzlibrary{intersections,calc,arrows.meta}

\usepackage{amsmath, amssymb, amsthm}
\usepackage{ascmac}
\usepackage{comment}
\usepackage{empheq}
\usepackage[shortlabels,inline]{enumitem}
\usepackage{fancybox}
\usepackage{fancyhdr}
\usepackage{here}
\usepackage{lastpage}
\usepackage{listings, jvlisting}
\usepackage{fixdif}

\usepackage{stmaryrd}
\usepackage[listings]{tcolorbox}
%\usepackage{ascolorbox}
\usepackage{titlesec}
\usepackage{ulem}
\usepackage{url}
\usepackage{verbatim}
\usepackage{wrapfig}
\usepackage{xcolor}
\usepackage{luatexja-ruby}
\usepackage{varwidth}
\usepackage[version=3]{mhchem}
\usepackage{wrapfig}


\usepackage{physics2}
	\usephysicsmodule{ab}
	\usephysicsmodule{ab.braket}
	\usephysicsmodule{ab.legacy}
	%\usephysicsmodule{braket}
	\usephysicsmodule{diagmat}
	\usephysicsmodule{xmat}
	\usephysicsmodule{nabla.legacy}
	\usephysicsmodule{qtext.legacy}

\usepackage[ISO]{diffcoeff}
\difdef { f, s } { D }
{ op-symbol = \mathrm{D} }


\newcommand{\mctext}[1]{\mbox{\textcircled{\scriptsize{#1}}}}
\newcommand{\ctext}[1]{\textcircled{\scriptsize{#1}}}
\newcommand{\ds}{\displaystyle}
\newcommand{\comb}[2]{{}_{#1}\mathrm{C}_{#2}}
\newcommand{\hs}{\hspace}
\newcommand{\vs}{\vspace}
\newcommand{\emphvs}{\vspace{1em}\notag\\}
\newcommand{\ora}{\overrightarrow}
\newcommand{\oramr}[1]{\overrightarrow{\mathrm{#1}}}
\newcommand{\tri}{\triangle}
\newcommand{\mr}{\mathrm}
\newcommand{\mb}{\mathbb}
\newcommand{\mrvec}[1]{\overrightarrow{\mathrm{#1}}}
\newcommand{\itvec}{\overrightarrow}
\newcommand{\bs}{\boldsymbol}
\newcommand{\ra}{\rightarrow}
\newcommand{\Ra}{\Rightarrow}
\newcommand{\lra}{\longrightarrow}
\newcommand{\Lra}{\Longrightarrow}
\newcommand{\la}{\leftarrow}
\newcommand{\La}{\Leftarrow}
\newcommand{\lla}{\longleftarrow}
\newcommand{\Lla}{\Longleftarrow}
\newcommand{\lr}{\leftrightarrow}
\newcommand{\llr}{\longleftrightarrow}
\newcommand{\Llr}{\Longleftrightarrow}
\renewcommand{\deg}{{}^\circ}
\newcommand{\phbox}{\fbox{\phantom{1\hspace{2em}}}}
\newcommand{\boxnum}[1]{\fbox{\phantom{\hspace{1em}}({#1})\phantom{\hspace{1em}}}}
\newcommand{\boxkana}[1]{\fbox{\phantom{\hspace{1em}}{#1}\phantom{\hspace{1em}}}}
\newcommand{\boxkm}[2]{\fbox{\, {#1}\phantom{\hspace{0.2em}} \,  ${#2}$}}
\newcommand{\hzw}{\hspace{1\zw}}

\renewcommand{\baselinestretch}{1.25}
\parindent=1\zw

%入97
\begin{document}
\noindent
\fbox{NewTH1-20} [同志社大2013]

次の文中の空欄ア〜クに当てはまる式を記せ.ただし,重力加速度の大きさを$g\, \text{〔m/s${}^2$〕}$とする.

図1のように,2つの小さなローラーAとBがあり,その軸は水平方向に$2l \, \text{〔m〕}$離れて平行に固定されている.A,B上に密度の一様な質量$m \, \text{〔kg〕}$の板をのせ,その運動について考える.A,Bそれぞれと板との接点を結ぶ線分の中点を原点Oとし,この線分に沿ってAからBへの向きを正として$x$軸をとる.板は薄く$x$方向の長さが$4l$の直方体で,A,Bそれぞれと板との間の動摩擦係数はともに$\mu'$とする.

最初に,Aが時計回りに,Bが反時計回りに高速で回転していて,それぞれ板の下面で常にすべっている状態を考える.板の重心Gは$x$軸の位置$\dfrac{l}{2}$にくるように板をローラーの上に静かにのせ,図2のように,Gが位置$x\, \text{〔m〕}$にきたとき,板がAとBから受ける動摩擦力の合力は$x$方向に\,\boxkana{ア}〔N〕である.この合力が$x$に比例し,その比例係数が負であることから,板は周期\,\boxkana{イ}〔s〕で単振動することがわかる.Gが原点Oにきたときの板の速さは\,\boxkana{ウ}〔m/s〕である.

次に,図3のように,AとBの距離を保ちながらBのほうを高くして,板と水平面とのなす角度が$\theta\,\text{〔rad〕}$となるようにした.Aを表面のなめらかなローラーA$'$に取りかえ,板との摩擦がなくなるようにした.Bは時計回りに高速で回転しており,板の下面で常にすべっている.板の重心Gの位置が$x$にあるとき,板がBから受ける摩擦力の大きさは\,\boxkana{エ}〔N〕となる.板を静かに置いたとき静止し続けるようなGの位置を$x_0\,\text{〔m〕}$とすると,この$x_0$が2つのローラーの間にあるのは$\mu'$と$\theta$の間に不等式$0 < \text{\boxkana{オ}} < \mu'$が成り立つときである.図3の場合,Gの位置が$x_0$から少しでもずれると,板は一方向に動き始め戻ってこない.

最後に,図3と同じように板が角度$\theta$だけ傾いた状態で,図4のように,BをなめらかなローラーB$'$に取りかえ,A$'$を元のAに戻して時計回りに高速で回転させる.板の重心Gが$x_1\,\text{〔m〕}$の位置にあるとき,板がAから受ける摩擦力は$x$方向に\,\boxkana{カ}〔N〕であり,Gが$-x_0$の位置になるように板を静かに置くと静止することがわかる.この位置からずれたとき,そのずれ$X = x_1 - (-x_0) = x_1 + x_0$を用いると,板に働く合力は$x$方向に$\text{\boxkana{キ}} \times X \,\text{〔N〕}$となり,$X$に比例し,その比例係数が負である.これより,$X$が適切な範囲にあれば.板は周期\,\boxkana{ク}〔s〕で単振動することがわかる.
\end{document}