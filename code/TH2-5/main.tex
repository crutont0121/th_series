%%%%%%%%%%%%%%%%%%%%%%%%%%
%%% author : Yamada. T %%%
%%% made for TH series %%%
%%%%%%%%%%%%%%%%%%%%%%%%%%

\documentclass[b5paper,10pt,fleqn] {ltjsarticle}

\usepackage[margin=10truemm]{geometry}

\usepackage{pict2e, graphicx}
\usepackage{tikz}
\usetikzlibrary{intersections,calc,arrows.meta}

\usepackage{amsmath, amssymb, amsthm}
\usepackage{ascmac}
\usepackage{comment}
\usepackage{empheq}
\usepackage[shortlabels,inline]{enumitem}
\usepackage{fancybox}
\usepackage{fancyhdr}
\usepackage{here}
\usepackage{lastpage}
\usepackage{listings, jvlisting}
\usepackage{fixdif}

\usepackage{stmaryrd}
\usepackage[listings]{tcolorbox}
%\usepackage{ascolorbox}
\usepackage{titlesec}
\usepackage{ulem}
\usepackage{url}
\usepackage{verbatim}
\usepackage{wrapfig}
\usepackage{xcolor}
\usepackage{luatexja-ruby}
\usepackage{varwidth}
\usepackage[version=3]{mhchem}
\usepackage{wrapfig}


\usepackage{physics2}
	\usephysicsmodule{ab}
	\usephysicsmodule{ab.braket}
	\usephysicsmodule{ab.legacy}
	%\usephysicsmodule{braket}
	\usephysicsmodule{diagmat}
	\usephysicsmodule{xmat}
	\usephysicsmodule{nabla.legacy}
	\usephysicsmodule{qtext.legacy}

\usepackage[ISO]{diffcoeff}
\difdef { f, s } { D }
{ op-symbol = \mathrm{D} }


\newcommand{\mctext}[1]{\mbox{\textcircled{\scriptsize{#1}}}}
\newcommand{\ctext}[1]{\textcircled{\scriptsize{#1}}}
\newcommand{\ds}{\displaystyle}
\newcommand{\comb}[2]{{}_{#1}\mathrm{C}_{#2}}
\newcommand{\hs}{\hspace}
\newcommand{\vs}{\vspace}
\newcommand{\emphvs}{\vspace{1em}\notag\\}
\newcommand{\ora}{\overrightarrow}
\newcommand{\ol}{\overline}
\newcommand{\oramr}[1]{\overrightarrow{\mathrm{#1}}}
\newcommand{\tri}{\triangle}
\newcommand{\mr}{\mathrm}
\newcommand{\mb}{\mathbb}
\newcommand{\mrvec}[1]{\overrightarrow{\mathrm{#1}}}
\newcommand{\itvec}{\overrightarrow}
\newcommand{\bs}{\boldsymbol}
\newcommand{\ra}{\rightarrow}
\newcommand{\Ra}{\Rightarrow}
\newcommand{\lra}{\longrightarrow}
\newcommand{\Lra}{\Longrightarrow}
\newcommand{\la}{\leftarrow}
\newcommand{\La}{\Leftarrow}
\newcommand{\lla}{\longleftarrow}
\newcommand{\Lla}{\Longleftarrow}
\newcommand{\lr}{\leftrightarrow}
\newcommand{\llr}{\longleftrightarrow}
\newcommand{\Llr}{\Longleftrightarrow}
\renewcommand{\deg}{{}^\circ}
\newcommand{\phbox}{\fbox{\phantom{1\hspace{2em}}}}
\newcommand{\boxnum}[1]{\fbox{\phantom{\hspace{1em}}({#1})\phantom{\hspace{1em}}}}
\newcommand{\boxkana}[1]{\fbox{\phantom{\hspace{1em}}{#1}\phantom{\hspace{1em}}}}
\newcommand{\boxkm}[2]{\fbox{\, {#1}\phantom{\hspace{0.2em}} \,  {#2}}}
\newcommand{\hzw}{\hspace{1\zw}}

\renewcommand{\baselinestretch}{1.25}
\parindent=1\zw

%%入115

\begin{document}
\noindent
\fbox{NewTH2-5} [東京工業大]

図1に示すように,断面積$S \,\text{〔m${}^2$〕}$の円筒状シリンダー密閉容器が,なめらかに動く質量$m\,\text{〔kg〕}$のピストンによりA室とB室に仕切られている.A室とB室にはそれぞれ気体を封入することができる.両室の気密性は高く,気体の漏れは無視できる.ピストンおよびシリンダーの側面と底面は熱を通さない.一方,シリンダーの上面は熱を通す.シリンダー各室内では温度と圧力は常に均一である.重力加速度の大きさを$g\, \text{〔m/s${}^2$〕}$,シリンダーに封入される理想気体の定積モル比熱を$C_v\, \text{〔J/(mol$\cdot$K)〕}$,気体定数を$R\, \text{〔J/(mol$\cdot$K)〕}$とし,以下の問いに答えよ.ただし,シリンダーに封入される理想気体の質量はピストンの質量に対し十分小さく無視できる.

\begin{enumerate}[I]
  \item {\hzw}まず,A室のみに1 molの理想気体を封入したシリンダーを水平な床に垂直に立てた.B室は真空である.ピストンはシリンダー上面から糸によりつるされた状態で静止しており,このときのA室内の気体の体積,温度,圧力は,それぞれ$2V_0 \,\text{〔m${}^3$〕}$,$T_0 \,\text{〔K〕}$,$p_0\,\text{〔Pa〕}$であった.B室の体積は$V_0 \,\text{〔m${}^3$〕}$であった.この状態を初期状態とよぶ.
  \begin{enumerate}[(1)]
    \item {\hzw}ピストンをつるしている糸を切断したところ,ピストンは気体の体積が$V_0$になるまで下方に移動し,その後は上方に向かう運動に転じた.ピストンが再下点に達したときの気体の温度を$T_1 \,\text{〔K〕}$とする.このときの気体の内部エネルギーの初期状態に対する変化量$\varDelta U_1\,\text{〔J〕}$を$T_1$,$T_0$,$C_v$を用いて表せ.
    \item {\hzw}ピストンが最下点に達したときのピストンの位置エネルギーの初期状態に対する変化量$\varDelta U_\mr{P}\, \text{〔J〕}$を$V_0$,$m$,$S$,$g$を用いて表せ.
    \item {\hzw}(1)と(2)を用いて$T_1$を求めよ.
  \end{enumerate}
  \item {\hzw}次に,B室にもA室と同じ理想気体を1 mol封入した.このシリンダーを図2に示すように,水平面内で回転できる円盤上に固定した.シリンダーの中心軸は円盤の回転軸に直交し,A室が円盤の外側を向いている.B室側のシリンダー端面には熱源を接続し,B室の気体が圧力を常に一定に保ちながら状態変化するように熱を供給する.円盤が静止しているときのA室の気体の体積,温度,圧力は,それぞれ$2V_0$,$T_0$,$p_0$であり,B室の気体の体積,温度,圧力は,それぞれ$V_0$,$\dfrac{T_0}{2}$,$p_0$であった.この状態を状態1とよぶ.円盤を静かに回転させ始めたところ,ピストンは静かに動き始め,その後,円盤の回転角速度を徐々に増し,ある回転角速度に達した後は等速回転させた.
  
  {\hzw}このとき,ピストンはA室とB室の気体の体積が,それぞれ$V_0$,$2V_0$となる位置で静止していた.これを状態2とよぶ.このA室とB室の気体の状態変化をシリンダーとともに回転する観測者が見るとして,以下の問いに答えよ.
  \begin{enumerate}[(1), resume]
    \item {\hzw}A室とB室の気体の状態変化の概略を,それぞれ図3の$p$--$V$グラフ上に描け.A室とB室の状態1,2をそれぞれ$\mr{A}_1$,$\mr{A}_2$,$\mr{B}_1$,$\mr{B}_2$として図中に示し,各状態における圧力と体積を明記すること.ただし,A室の気体の状態2における圧力として$p_2 \,\text{〔Pa〕}$を用いてよい.なお,図3には,1 molの理想気体の温度$T_0$,$\dfrac{T_0}{2}$における等温変化の曲線が記入されている.これらの曲線との関係も考慮して記入すること.さらに,円盤の回転によりピストンにはたらく遠心力がA室の気体にした仕事に対応する領域を斜線で示せ.
    \item {\hzw}ピストンにはたらく遠心力がA室の気体にした仕事$W_\mr{C}\,\text{〔J〕}$を求めよ.ただし,Iの結果を用いてもよい.
  \end{enumerate}
\end{enumerate}
\end{document}