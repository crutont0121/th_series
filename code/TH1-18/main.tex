%%%%%%%%%%%%%%%%%%%%%%%%%%
%%% author : Yamada. T %%%
%%% made for TH series %%%
%%%%%%%%%%%%%%%%%%%%%%%%%%

\documentclass[b5paper,10pt,fleqn] {ltjsarticle}

\usepackage[margin=10truemm]{geometry}

\usepackage{pict2e, graphicx}
\usepackage{tikz}
\usetikzlibrary{intersections,calc,arrows.meta}

\usepackage{amsmath, amssymb, amsthm}
\usepackage{ascmac}
\usepackage{comment}
\usepackage{empheq}
\usepackage[shortlabels,inline]{enumitem}
\usepackage{fancybox}
\usepackage{fancyhdr}
\usepackage{here}
\usepackage{lastpage}
\usepackage{listings, jvlisting}
\usepackage{fixdif}

\usepackage{stmaryrd}
\usepackage[listings]{tcolorbox}
%\usepackage{ascolorbox}
\usepackage{titlesec}
\usepackage{ulem}
\usepackage{url}
\usepackage{verbatim}
\usepackage{wrapfig}
\usepackage{xcolor}
\usepackage{luatexja-ruby}
\usepackage{varwidth}
\usepackage[version=3]{mhchem}
\usepackage{wrapfig}


\usepackage{physics2}
	\usephysicsmodule{ab}
	\usephysicsmodule{ab.braket}
	\usephysicsmodule{ab.legacy}
	%\usephysicsmodule{braket}
	\usephysicsmodule{diagmat}
	\usephysicsmodule{xmat}
	\usephysicsmodule{nabla.legacy}
	\usephysicsmodule{qtext.legacy}

\usepackage[ISO]{diffcoeff}
\difdef { f, s } { D }
{ op-symbol = \mathrm{D} }


\newcommand{\mctext}[1]{\mbox{\textcircled{\scriptsize{#1}}}}
\newcommand{\ctext}[1]{\textcircled{\scriptsize{#1}}}
\newcommand{\ds}{\displaystyle}
\newcommand{\comb}[2]{{}_{#1}\mathrm{C}_{#2}}
\newcommand{\hs}{\hspace}
\newcommand{\vs}{\vspace}
\newcommand{\emphvs}{\vspace{1em}\notag\\}
\newcommand{\ora}{\overrightarrow}
\newcommand{\ol}{\overline}
\newcommand{\oramr}[1]{\overrightarrow{\mathrm{#1}}}
\newcommand{\tri}{\triangle}
\newcommand{\mr}{\mathrm}
\newcommand{\mb}{\mathbb}
\newcommand{\mrvec}[1]{\overrightarrow{\mathrm{#1}}}
\newcommand{\itvec}{\overrightarrow}
\newcommand{\bs}{\boldsymbol}
\newcommand{\ra}{\rightarrow}
\newcommand{\Ra}{\Rightarrow}
\newcommand{\lra}{\longrightarrow}
\newcommand{\Lra}{\Longrightarrow}
\newcommand{\la}{\leftarrow}
\newcommand{\La}{\Leftarrow}
\newcommand{\lla}{\longleftarrow}
\newcommand{\Lla}{\Longleftarrow}
\newcommand{\lr}{\leftrightarrow}
\newcommand{\llr}{\longleftrightarrow}
\newcommand{\Llr}{\Longleftrightarrow}
\renewcommand{\deg}{{}^\circ}
\newcommand{\phbox}{\fbox{\phantom{1\hspace{2em}}}}
\newcommand{\boxnum}[1]{\fbox{\phantom{\hspace{1em}}({#1})\phantom{\hspace{1em}}}}
\newcommand{\boxkana}[1]{\fbox{\phantom{\hspace{1em}}{#1}\phantom{\hspace{1em}}}}
\newcommand{\boxkm}[2]{\fbox{\, {#1}\phantom{\hspace{0.2em}} \,  {#2}}}
\newcommand{\hzw}{\hspace{1\zw}}

\renewcommand{\baselinestretch}{1.25}
\parindent=1\zw

%% TH3-2
\begin{document}
\noindent
\fbox{NewTH1-18} [東京工業大2021]

水平な床の面に座標軸$x, y$をとり,その上で大きさが無視できる質量$m$の3つの小球A,B,Cを,長さ$L$の2本の糸でB--A--Cの順につないだものを滑らせる実験を行う.意図は伸び縮みせず,その質量は無視でき,床と小球の間に摩擦はないものとする.また,ゆかは十分広く,運動の途中で小球が床の端に達することはない.

\begin{enumerate}[label={〔\Alph*〕}]
  \item {\hzw}小球Aを原点に,小球Bと小球Cを$y$軸上の$y = L$と$y = -L$の位置に,それぞれ静止させる.時刻$t = 0$において,図1のように,小球Aにのみ$x$軸の正の向きに速さ$V_0$を与えて運動を開始させた.その後の小球の運動を観察したところ,運動開始直後は小球Bと小球Cの速度は0であり,その後小球Bと小球Cは近づいていき,やがて$x$軸上のある点で衝突した.運動の開始から衝突までの間,糸がたるむことはなく,小球Aから見ると,小球Bと小球Cは小球Aを中心とする円運動をした.以下の問に答えよ.
  \begin{enumerate}[(a)]
    \item {\hzw}小球Bと小球Cが衝突する直前における,小球Bの速度の$x$成分$V_x$を$V_0$を用いて表せ.
    \item {\hzw}小球Bと小球Cが衝突する直前における,小球Bの速度の$y$成分$V_y$を$V_0$を用いて表せ.
    \item {\hzw}運動開始直後における,小球Bにつながれた糸の張力の大きさ$T$を$V_0, m, L$を用いて表せ.
    \item {\hzw}小球Bと小球Cが衝突する直前における,小球Bにつながれた糸の張力の大きさ$T'$を$V_0, m, L$を用いて表せ.
  \end{enumerate}
  \item {\hzw}図2に示すように,小球Aを原点に,小球Bと小球Cをそれぞれ座標$\ds \ab(-L\cos\theta, L\sin\theta)$と$\ds \ab(-L\cos\theta, -L\sin\theta)\, ( 0\deg \leqq \theta <  90 \deg)$の点に配置し,静止させる.時刻$t \geqq 0$において,小球Aに$x$軸の正の向きに一定の大きさ$F$の力を加える.以下のように,$\theta$の値を変えて実験1と実験2を行い,小球A, B,Cの運動を記録した.いずれの場合にも糸がたるむことはなかった.
  
  \begin{description}[leftmargin=4\zw]
    \item[\rm 実験1:] $\theta = 0\deg$となるように,すなわち,小球Bと小球Cは接するように,小球を配置し静止させる.そして$t \geqq 0$において小球Aに$x$軸の正の向きに一定の大きさ$F$の力を加えたところ,小球Bと小球Cは接したまま,3つの小球は$x$軸の正の向きに同じ加速度で等加速度運動した.その加速度の大きさは$a_1$であった.
    \vspace{1em}
    \item[\rm 実験2:] $\theta$をある値$\theta_2\, (0\deg < \theta_2 < 90 \deg)$にとり,$t \geqq 0$において小球Aに$x$軸の正の向きに一定の大きさ$F$の力を加えたところ,小球Bと小球Cは時刻$t_2$においてはじめて衝突した.衝突直前の小球Bと小球Cの速度ベクトルのなす角は$60 \deg$であった.
  \end{description}

  {\hzw}図3は実験1と実験2における小球Aの$x$座標の時間変化を$0 \leqq t \leqq t_2$においてグラフにしたものである.ただし,グラフは概形である.

  {\hzw}これらの実験における小球の運動に関する,以下の問に答えよ.

  \begin{enumerate}[(a), resume]
    \item {\hzw}実験1における小球Aの加速度の大きさ$a_1$を$m$と$F$を用いて表せ.
    \item {\hzw}時刻$t_2$における実験1の小球の速さを$v$とする.実験2の小球Bの衝突直前における速さ$w$を$v$を用いて表せ.
    \item {\hzw}時刻$t_2$における実験1と実験2の小球Aの$x$座標をそれぞれ$x_1$および$x_2$とする.比$\dfrac{x_2}{x_1}$を求めよ.
    \item {\hzw}実験2の$0 < t < t_2$における小球Aの加速度の大きさ$a_2$のグラフの概形として最も適当なものを図4の(ア)〜(シ)のうちから選び,記号で答えよ.
  \end{enumerate}
\end{enumerate}


\end{document}