%%%%%%%%%%%%%%%%%%%%%%%%%%
%%% author : Yamada. T %%%
%%% made for TH series %%%
%%%%%%%%%%%%%%%%%%%%%%%%%%

\documentclass[b5paper,10pt,fleqn] {ltjsarticle}

\usepackage[margin=10truemm]{geometry}

\usepackage{pict2e, graphicx}
\usepackage{tikz}
\usetikzlibrary{intersections,calc,arrows.meta}

\usepackage{amsmath, amssymb, amsthm}
\usepackage{ascmac}
\usepackage{comment}
\usepackage{empheq}
\usepackage[shortlabels,inline]{enumitem}
\usepackage{fancybox}
\usepackage{fancyhdr}
\usepackage{here}
\usepackage{lastpage}
\usepackage{listings, jvlisting}
\usepackage{fixdif}

\usepackage{stmaryrd}
\usepackage[listings]{tcolorbox}
%\usepackage{ascolorbox}
\usepackage{titlesec}
\usepackage{ulem}
\usepackage{url}
\usepackage{verbatim}
\usepackage{wrapfig}
\usepackage{xcolor}
\usepackage{luatexja-ruby}
\usepackage{varwidth}
\usepackage[version=3]{mhchem}
\usepackage{wrapfig}


\usepackage{physics2}
	\usephysicsmodule{ab}
	\usephysicsmodule{ab.braket}
	\usephysicsmodule{ab.legacy}
	%\usephysicsmodule{braket}
	\usephysicsmodule{diagmat}
	\usephysicsmodule{xmat}
	\usephysicsmodule{nabla.legacy}
	\usephysicsmodule{qtext.legacy}

\usepackage[ISO]{diffcoeff}
\difdef { f, s } { D }
{ op-symbol = \mathrm{D} }


\newcommand{\mctext}[1]{\mbox{\textcircled{\scriptsize{#1}}}}
\newcommand{\ctext}[1]{\textcircled{\scriptsize{#1}}}
\newcommand{\ds}{\displaystyle}
\newcommand{\comb}[2]{{}_{#1}\mathrm{C}_{#2}}
\newcommand{\hs}{\hspace}
\newcommand{\vs}{\vspace}
\newcommand{\emphvs}{\vspace{1em}\notag\\}
\newcommand{\ora}{\overrightarrow}
\newcommand{\oramr}[1]{\overrightarrow{\mathrm{#1}}}
\newcommand{\tri}{\triangle}
\newcommand{\mr}{\mathrm}
\newcommand{\mb}{\mathbb}
\newcommand{\mrvec}[1]{\overrightarrow{\mathrm{#1}}}
\newcommand{\itvec}{\overrightarrow}
\newcommand{\bs}{\boldsymbol}
\newcommand{\ra}{\rightarrow}
\newcommand{\Ra}{\Rightarrow}
\newcommand{\lra}{\longrightarrow}
\newcommand{\Lra}{\Longrightarrow}
\newcommand{\la}{\leftarrow}
\newcommand{\La}{\Leftarrow}
\newcommand{\lla}{\longleftarrow}
\newcommand{\Lla}{\Longleftarrow}
\newcommand{\lr}{\leftrightarrow}
\newcommand{\llr}{\longleftrightarrow}
\newcommand{\Llr}{\Longleftrightarrow}
\renewcommand{\deg}{{}^\circ}
\newcommand{\phbox}{\fbox{\phantom{1\hspace{2em}}}}
\newcommand{\boxnum}[1]{\fbox{\phantom{\hspace{1em}}({#1})\phantom{\hspace{1em}}}}
\newcommand{\boxkana}[1]{\fbox{\phantom{\hspace{1em}}{#1}\phantom{\hspace{1em}}}}
\newcommand{\boxkm}[2]{\fbox{\, {#1}\phantom{\hspace{0.2em}} \,  ${#2}$}}
\newcommand{\hzw}{\hspace{1\zw}}

\renewcommand{\baselinestretch}{1.25}
\parindent=1\zw


%%入93
\begin{document}
\noindent
\fbox{NewTH1-19} [大阪大2006]

図のように,大きさが無視できる質量$m$の物体が,水平なばねで壁とつながれ,回転ベルトの水平部分の上に置かれている.
ばねは十分に長い自然の長さ$d$をもち,ばね定数は$k$である.
物体とベルトの間の静止摩擦係数は$\mu_1$,動摩擦係数は$\mu_2$で$\mu_1 > \mu_2$である.
回転ベルトの上側のベルトは,図のように壁に向かって一定の速さ$w$で動かすことができる(以後,回転ベルトの上側のベルトを単にベルトとよぶ).$w$の大きさを変化させたとき,$w$のある値を境にして物体の運動の様子が大きく変化するのが観測された.物体の静止した壁に対する(壁から見た)運動と,動いているベルトに対する(ベルトから見た)相対速度とに注意しながら,この変化について考察しよう.
壁面を原点とする物体の位置座標を$x$,図の右向きを$x$の正の向きとし,重力加速度の大きさを$g$として,以下の記述の中の問いに答えよ.

\begin{enumerate}[A]
  \item {\hzw}まず,ベルトが静止した状態($w = 0$)で,$x = d$の位置に物体をベルトの上に静止させて置いた.次に,小さな$w$でベルトを動かした.すると,物体はベルトに運ばれて壁に近づき,ある位置$x = x_0$でベルトに対して右向きに静かに(ベルトに対する相対的な初速度0で)すべり出した.
  \begin{enumerate}[(1)]
    \item {\hzw}$x_0$を$d$,$m$,$k$,$\mu_1$,$\mu_2$,$w$,$g$の中から必要なものを用いて表せ.
  \end{enumerate}
  \item {\hzw}今度は,まず物体を上で決めた$x = x_0$の位置まであらかじめ手で移し,一定の速さ$w$で動いているベルトの上で静かに手をはなした.すると,物体は壁に対する初速度0で右向きにすべり出した.この後の物体の運動を見てみよう.
  \begin{enumerate}[(1), resume]
    \item {\hzw}物体が右向きにすべっているとき,物体の位置を$x$,加速度を$a$(右向きを正の向きとする)として,物体の運動方程式を求めよ.
  \end{enumerate}
  {\hzw}この運動方程式は,ある見かけの自然の長さ$L$,ばね定数$k$の水平なばねで壁につながれた質量$m$の物体の,摩擦のない水平面上での運動の運動方程式と同じである.したがって,物体の運動をこのような単純な状況での運動に置きかえて考えることができる.
  \begin{enumerate}[(1), resume]
    \item {\hzw}$L$を$d$,$m$,$k$,$\mu_1$,$\mu_2$,$w$,$g$の中から必要なものを用いて表せ.
    \item {\hzw}物体が壁から最も離れる位置(最大の伸びの位置)を$x = x_\mr{M}$として,$x_\mr{M}$を$L$と$x_0$を用いて表せ.
  \end{enumerate}
  {\hzw}$x = x_\mr{M}$の位置では物体は壁に対して静止するが,このときもベルトは壁に向かって動いている.したがって,物体のベルトに対する相対的な運動の向きは変わらず,物体にはたらく摩擦力の向きも変わらない.この後,物体は壁の向きに加速され,ベルトと同じ速さになったところでベルトに対して静止する.その瞬間まで,物体は上で述べた摩擦のない水平面上での運動方程式にしたがって運動する.
  \begin{enumerate}[(1), resume]
    \item {\hzw}ベルトに対して静止する位置の$x$の値を$L$,$x_\mr{M}$,$m$,$k$,$w$を用いて表せ.
  \end{enumerate}
  {\hzw}物体がいったんベルトに対して静止すると,摩擦は動摩擦から静止摩擦に変わるので,この後,物体はベルトに対して静止したままベルトに運ばれて壁に近づき,$x = x_0$で再びすべり出す.
  \item {\hzw}次に,$w$を少しずつ増加させながら,同様に$x = x_0$で手をはなす実験を繰り返した.すると$w$がある値$w_\mr{c}$を超えたとき,物体はベルトに対して静止することなく単振動を続けるようになった.
  \begin{enumerate}[(1), resume]
    \item {\hzw}$w_\mr{c}$を$m$,$k$,$\mu_1$,$\mu_2$,$g$を用いて表せ.
  \end{enumerate}
  {\hzw}このように$w$が十分に大きいときには,ベルトと物体の間に摩擦があるにもかかわらず,物体は単振動を継続する.
\end{enumerate}
\end{document}