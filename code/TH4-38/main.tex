%%%%%%%%%%%%%%%%%%%%%%%%%%
%%% author : Yamada. T %%%
%%% made for TH series %%%
%%%%%%%%%%%%%%%%%%%%%%%%%%

\documentclass[b5paper,10pt,fleqn] {ltjsarticle}

\usepackage[margin=10truemm]{geometry}

\usepackage{pict2e, graphicx}
\usepackage{tikz}
\usetikzlibrary{intersections,calc,arrows.meta}

\usepackage{amsmath, amssymb, amsthm}
\usepackage{ascmac}
\usepackage{comment}
\usepackage{empheq}
\usepackage[shortlabels,inline]{enumitem}
\usepackage{fancybox}
\usepackage{fancyhdr}
\usepackage{here}
\usepackage{lastpage}
\usepackage{listings, jvlisting}
\usepackage{fixdif}

\usepackage{stmaryrd}
\usepackage[listings]{tcolorbox}
%\usepackage{ascolorbox}
\usepackage{titlesec}
\usepackage{ulem}
\usepackage{url}
\usepackage{verbatim}
\usepackage{wrapfig}
\usepackage{xcolor}
\usepackage{luatexja-ruby}
\usepackage{varwidth}
\usepackage[version=3]{mhchem}
\usepackage{wrapfig}


\usepackage{physics2}
	\usephysicsmodule{ab}
	\usephysicsmodule{ab.braket}
	\usephysicsmodule{ab.legacy}
	%\usephysicsmodule{braket}
	\usephysicsmodule{diagmat}
	\usephysicsmodule{xmat}
	\usephysicsmodule{nabla.legacy}
	\usephysicsmodule{qtext.legacy}

\usepackage[ISO]{diffcoeff}
\difdef { f, s } { D }
{ op-symbol = \mathrm{D} }


\newcommand{\mctext}[1]{\mbox{\textcircled{\scriptsize{#1}}}}
\newcommand{\ctext}[1]{\textcircled{\scriptsize{#1}}}
\newcommand{\ds}{\displaystyle}
\newcommand{\comb}[2]{{}_{#1}\mathrm{C}_{#2}}
\newcommand{\hs}{\hspace}
\newcommand{\vs}{\vspace}
\newcommand{\emphvs}{\vspace{1em}\notag\\}
\newcommand{\ora}{\overrightarrow}
\newcommand{\ol}{\overline}
\newcommand{\oramr}[1]{\overrightarrow{\mathrm{#1}}}
\newcommand{\tri}{\triangle}
\newcommand{\mr}{\mathrm}
\newcommand{\mb}{\mathbb}
\newcommand{\mrvec}[1]{\overrightarrow{\mathrm{#1}}}
\newcommand{\itvec}{\overrightarrow}
\newcommand{\bs}{\boldsymbol}
\newcommand{\ra}{\rightarrow}
\newcommand{\Ra}{\Rightarrow}
\newcommand{\lra}{\longrightarrow}
\newcommand{\Lra}{\Longrightarrow}
\newcommand{\la}{\leftarrow}
\newcommand{\La}{\Leftarrow}
\newcommand{\lla}{\longleftarrow}
\newcommand{\Lla}{\Longleftarrow}
\newcommand{\lr}{\leftrightarrow}
\newcommand{\llr}{\longleftrightarrow}
\newcommand{\Llr}{\Longleftrightarrow}
\renewcommand{\deg}{{}^\circ}
\newcommand{\phbox}{\fbox{\phantom{1\hspace{2em}}}}
\newcommand{\boxnum}[1]{\fbox{\phantom{\hspace{1em}}({#1})\phantom{\hspace{1em}}}}
\newcommand{\boxkana}[1]{\fbox{\phantom{\hspace{1em}}{#1}\phantom{\hspace{1em}}}}
\newcommand{\boxkm}[2]{\fbox{\, {#1}\phantom{\hspace{0.2em}} \,  {#2}}}
\newcommand{\hzw}{\hspace{1\zw}}

\renewcommand{\baselinestretch}{1.25}
\parindent=1\zw

%TH1-8

\begin{document}
\noindent
\fbox{NewTH4-38} [京都大]


次の文を読んで,空欄に適した式を記せ. 

図1に示すように,水平面上に間隔$l$で互いに平行に配置された2本の導体レールの上に,これらと直角に質量$m$の導体棒MNが置かれており,それがばね定数$k$の不導体のばねにつながれている.
導体棒MNは,レールと平行にその上を左右に摩擦なく動けるものとする.
また,導体棒MNの動く範囲には,紙面に垂直に一様な磁束密度$B$の静磁場が,紙面の裏から表の向きに常に作用している.
なお,図1に示したように$x$座標をとり,導体棒MNが$x = 0$の位置にあるとき,ばねは自然の長さであるとする.

%% fig

ここで,レールの右端PとQの間に,図2に示すようないろいろな素子をつないで1つのループ回路PMMQを構成したときに,導体棒MNがどのような運動をするかについて考える.
ただし,レールと導体棒MNおよびその接触点の電気抵抗,ならびに2本のレール間の電気容量は無視できるものとし,また,ループ回路に流れる電流の作る磁場は,静電場$B$や,PとQの間につないだ素子に影響を与えないものとする.

\begin{enumerate}[I]  
  \item いま,ばねを引き伸ばして,導体棒MNを$x = a$の位置までゆっくりと動かし,静かに放す.
     P,Q間を開放したままのとき,導体棒MNは角振動数$\omega _0 = \text{\boxnum{1}}$の単振動をする.導体棒MNを放した時点を時刻$t = 0$とすると,P,Q間に生じる誘導起電力の大きさは,$t$の関数として,\boxnum{2}と表せる. 
  \item P,Q間に,図2の(a)に示した電気抵抗$R$の抵抗をつないだ状態で,導体棒MNを$x = a$の位置までゆっくりと動かし,静かに放したところ,導体棒MNはP,Q間を開放していたときと同じような振動をしながら,少しずつその振幅が小さくなり,十分な時間が経過したのち,$x = 0$の位置に静止した.導体棒MNを放してから静止するまでの間に,電気抵抗$R$で消費されたエネルギーは\boxnum{3}である.この後,P,Q間に,(a)にかえて,(b)に示した電気抵抗$R$の抵抗と起電力$E$の電池からなる素子を図の向きにつなぐと,導体棒MNは,また振動をはじめ,少しずつその振幅が小さくなって,十分な時間が経過したのちには,$x = \text{\boxnum{4}}$の位置に静止した.
    静止した後,電気抵抗$R$で単位時間に消費されるエネルギーは\boxnum{5}である. 
  \item はじめに戻って,P,Q間を開放した状態で,導体棒MNを$x = a$の位置までゆっくりと動かし,P,Q間に,図2の(c)に示した電気容量$C$のコンデンサーをつないで,導体棒MNを静かに放すと,角振動数$\omega_1$で単振動した.導体棒MNを放した時点を時刻$t = 0$とすると,コンデンサーに蓄えられる電荷の量は,$t$の関数として,\boxnum{6}と表せる(ただし,図1において,コンデンサーの上側の電極に正電荷がたまる場合を正とする).導体棒MNが$x = 0$を通過する瞬間にコンデンサーに蓄えられているエネルギー\boxnum{7}と,そのときの運動エネルギーの和は,$t = 0$の時点に与えた全エネルギーに等しいので,結局,この単振動の角振動数は,$\omega_1 = \text{\boxnum{8}}$であることがわかる.
  \item 今度は,P,Q間に,図2の(d)に示した自己インダクタンス$L$のコイルをつないだ回路において,導体棒MNが,$x = 0$を中心とした角振動数$\omega_2$,振幅$a$の単振動をしているとする.$x= 0$を右に通過するある時点を$t = 0$と定め,そのとき回路を流れる電流は$0$であったとする.回路に流れる電流は,$t$の関数として,\boxnum{9}と表せる(ただし,図1の$i$の向きを正とする).$x = a$の位置においてコイルに蓄えられているエネルギー\boxnum{10}と,そのときにばねに蓄えられているエネルギーの和は,$t = 0$の時点の全エネルギーと等しいので,結局,この単振動の角振動数は$\omega_2 = \text{\boxnum{11}}$であることがわかる.

\end{enumerate}

\end{document}
