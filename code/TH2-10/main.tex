%%%%%%%%%%%%%%%%%%%%%%%%%%
%%% author : Yamada. T %%%
%%% made for TH series %%%
%%%%%%%%%%%%%%%%%%%%%%%%%%

\documentclass[b5paper,10pt,fleqn] {ltjsarticle}

\usepackage[margin=10truemm]{geometry}

\usepackage{pict2e, graphicx}
\usepackage{tikz}
\usetikzlibrary{intersections,calc,arrows.meta}

\usepackage{amsmath, amssymb, amsthm}
\usepackage{ascmac}
\usepackage{comment}
\usepackage{empheq}
\usepackage[shortlabels,inline]{enumitem}
\usepackage{fancybox}
\usepackage{fancyhdr}
\usepackage{here}
\usepackage{lastpage}
\usepackage{listings, jvlisting}
\usepackage{fixdif}

\usepackage{stmaryrd}
\usepackage[listings]{tcolorbox}
%\usepackage{ascolorbox}
\usepackage{titlesec}
\usepackage{ulem}
\usepackage{url}
\usepackage{verbatim}
\usepackage{wrapfig}
\usepackage{xcolor}
\usepackage{luatexja-ruby}
\usepackage{varwidth}
\usepackage[version=3]{mhchem}
\usepackage{wrapfig}


\usepackage{physics2}
	\usephysicsmodule{ab}
	\usephysicsmodule{ab.braket}
	\usephysicsmodule{ab.legacy}
	%\usephysicsmodule{braket}
	\usephysicsmodule{diagmat}
	\usephysicsmodule{xmat}
	\usephysicsmodule{nabla.legacy}
	\usephysicsmodule{qtext.legacy}

\usepackage[ISO]{diffcoeff}
\difdef { f, s } { D }
{ op-symbol = \mathrm{D} }


\newcommand{\mctext}[1]{\mbox{\textcircled{\scriptsize{#1}}}}
\newcommand{\ctext}[1]{\textcircled{\scriptsize{#1}}}
\newcommand{\ds}{\displaystyle}
\newcommand{\comb}[2]{{}_{#1}\mathrm{C}_{#2}}
\newcommand{\hs}{\hspace}
\newcommand{\vs}{\vspace}
\newcommand{\emphvs}{\vspace{1em}\notag\\}
\newcommand{\ora}{\overrightarrow}
\newcommand{\oramr}[1]{\overrightarrow{\mathrm{#1}}}
\newcommand{\tri}{\triangle}
\newcommand{\mr}{\mathrm}
\newcommand{\mb}{\mathbb}
\newcommand{\mrvec}[1]{\overrightarrow{\mathrm{#1}}}
\newcommand{\itvec}{\overrightarrow}
\newcommand{\bs}{\boldsymbol}
\newcommand{\ra}{\rightarrow}
\newcommand{\Ra}{\Rightarrow}
\newcommand{\lra}{\longrightarrow}
\newcommand{\Lra}{\Longrightarrow}
\newcommand{\la}{\leftarrow}
\newcommand{\La}{\Leftarrow}
\newcommand{\lla}{\longleftarrow}
\newcommand{\Lla}{\Longleftarrow}
\newcommand{\lr}{\leftrightarrow}
\newcommand{\llr}{\longleftrightarrow}
\newcommand{\Llr}{\Longleftrightarrow}
\renewcommand{\deg}{{}^\circ}
\newcommand{\phbox}{\fbox{\phantom{1\hspace{2em}}}}
\newcommand{\boxnum}[1]{\fbox{\phantom{\hspace{1em}}({#1})\phantom{\hspace{1em}}}}
\newcommand{\boxkana}[1]{\fbox{\phantom{\hspace{1em}}{#1}\phantom{\hspace{1em}}}}
\newcommand{\boxkm}[2]{\fbox{\, {#1}\phantom{\hspace{0.2em}} \,  ${#2}$}}
\newcommand{\hzw}{\hspace{1\zw}}

\renewcommand{\baselinestretch}{1.25}
\parindent=1\zw


%%入試119

\begin{document}
\noindent\fbox{NewTH2-10} [京都大]

熱力学は気体だけではなく,様々な対象にも適用することができる.
本問ではひも状の物体の熱力学を考えてみよう.
あるひも状の物体を引き伸ばし,長さが$L_\mr{min}$から$L_\mr{max}$の範囲内で張力$X$を測定したところ,$X$は長さ$L$に依存せず,絶対温度$T$および正の定数$A$を用いて$X = AT$と表された.
この物体の変形としては,$L$が$L_\mr{min}$から$L_\mr{max}$の範囲内にある1次元的な伸縮のみを考え,
また,内部エネルギー$U$は正の定数$C$を用いて$U = CT$となるとして,以下の問いに答えよ.

\begin{enumerate}[I, leftmargin=2\zw]
  \item この物体に外から微小仕事$\varDelta W$を加えて微小量$\varDelta L$だけ伸ばしたときに,$\varDelta W = X \varDelta L$という関係式が成り立つ.吸熱量を$\varDelta Q$,内部エネルギーの変化を$\varDelta U$としたとき,熱力学第1法則より$\varDelta U$は,$\varDelta Q$,$A$,$T$,$\varDelta L$を用いて,$\varDelta U = \text{\boxkana{ア}}$と表される.
    一方,$U = CT$より,物体を伸ばしたときの温度変化$\varDelta T$を用いて,内部エネルギーの変化は$\varDelta U = \text{\boxkana{イ}}$とも書ける.

    断熱的に物体をゆっくりと微小量$\varDelta L$伸ばしたときの温度変化$\varDelta T$は,$C$,$A$,$T$,$\varDelta L$を用いて表すと\boxkana{ウ}となり,温度は{エ\hs{1\zw} \ctext{1}下降する\hs{1\zw}\ctext{2}変わらない\hs{1\zw}\ctext{3}上昇する}.ただし,$\varDelta L > 0$とする.
  \item さて,この物体を断熱的にゆっくりと伸ばした.そのとき$L - \dfrac{C}{A} \log T$が一定であった.ここで,$\log T$は$T$の自然対数である.
    \begin{enumerate}[(1)]
      \item この理由を述べよ.ただし,正の定数$T$を$T + \varDelta T$までわずかに変化させたときの$\log T$の変化量を$\varDelta \log T$と表すと,$\dfrac{\varDelta \log T}{\varDelta T} = \dfrac{1}{T}$が成り立つことを用いてよい.
    \end{enumerate}
  \item 次に,同じ物体を温度$T$に保ったまま,長さ$L_0$から$L$までゆっくりと変化させたときに物体に外から加えられた仕事は\boxkana{オ}であり,その間の吸熱量は\boxkana{カ}である.ただし,\boxkana{オ}および\boxkana{カ}は$A$,$T$,$L_0$,$L$のみで表すこと.
  \item
    %%\begin{figure}[H]
    %%  \centering
    %%  \includegraphics[width=8cm]{fig/fig_2_10.pdf}
    %%\end{figure}
    図のように横軸を物体の長さ$L$とし,縦軸を温度$T$として,この物体の状態変化を表す.物体を温度$T_2$に保ちゆっくりと等温変化させ,その後ゆっくりと$T_1$まで断熱変化させ,さらに温度$T_1$でゆっくりと等温変化をさせた後に,断熱的にゆっくりと温度$T_2$の初めの状態に戻すサイクルを考えよう.
    高温熱源(温度$T_2$)から熱を吸収して仕事をし,低温熱源(温度$T_1$)に熱を放出するようなサイクルは,{キ\hs{1\zw}\ctext{1}(a)を時計回りに回る\hs{1\zw}\ctext{2}(a)を反時計回りに回る\hs{1\zw}\ctext{3}(b)を時計回りに回る\hs{1\zw}\ctext{4}(b)を反時計回りに回る}.
    \begin{enumerate}[(1), resume]
      \item このサイクルでは,$L_\mr{C} - L_\mr{B}$と$L_\mr{D} - L_\mr{A}$が等しくなる.その理由を述べよ.
    \end{enumerate}
  \item 一般にサイクルでの熱効率は,物体がサイクルを通じて外にする正味の仕事を,高温熱源から吸収する熱量$Q_\mr{in}$で割った量として導入される.よって,サイクルを動かす間の熱効率は,$Q_\mr{in}$とサイクルを動かす間に放出する熱量$Q_\mr{out}$を用いて\boxkana{ク}と書ける.
    \begin{enumerate}[(1), resume]
      \item これまでの結果を用いて,IVのサイクルの熱効率が$1 - \dfrac{T_1}{T_2}$となる理由を説明せよ.
    \end{enumerate}
\end{enumerate}

\end{document}
