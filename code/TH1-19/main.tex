%%%%%%%%%%%%%%%%%%%%%%%%%%
%%% author : Yamada. T %%%
%%% made for TH series %%%
%%%%%%%%%%%%%%%%%%%%%%%%%%

\documentclass[b5paper,10pt,fleqn] {ltjsarticle}

\usepackage[margin=10truemm]{geometry}

\usepackage{pict2e, graphicx}
\usepackage{tikz}
\usetikzlibrary{intersections,calc,arrows.meta}

\usepackage{amsmath, amssymb, amsthm}
\usepackage{ascmac}
\usepackage{comment}
\usepackage{empheq}
\usepackage[shortlabels,inline]{enumitem}
\usepackage{fancybox}
\usepackage{fancyhdr}
\usepackage{here}
\usepackage{lastpage}
\usepackage{listings, jvlisting}
\usepackage{fixdif}

\usepackage{stmaryrd}
\usepackage[listings]{tcolorbox}
%\usepackage{ascolorbox}
\usepackage{titlesec}
\usepackage{ulem}
\usepackage{url}
\usepackage{verbatim}
\usepackage{wrapfig}
\usepackage{xcolor}
\usepackage{luatexja-ruby}
\usepackage{varwidth}
\usepackage[version=3]{mhchem}
\usepackage{wrapfig}


\usepackage{physics2}
	\usephysicsmodule{ab}
	\usephysicsmodule{ab.braket}
	\usephysicsmodule{ab.legacy}
	%\usephysicsmodule{braket}
	\usephysicsmodule{diagmat}
	\usephysicsmodule{xmat}
	\usephysicsmodule{nabla.legacy}
	\usephysicsmodule{qtext.legacy}

\usepackage[ISO]{diffcoeff}
\difdef { f, s } { D }
{ op-symbol = \mathrm{D} }


\newcommand{\mctext}[1]{\mbox{\textcircled{\scriptsize{#1}}}}
\newcommand{\ctext}[1]{\textcircled{\scriptsize{#1}}}
\newcommand{\ds}{\displaystyle}
\newcommand{\comb}[2]{{}_{#1}\mathrm{C}_{#2}}
\newcommand{\hs}{\hspace}
\newcommand{\vs}{\vspace}
\newcommand{\emphvs}{\vspace{1em}\notag\\}
\newcommand{\ora}{\overrightarrow}
\newcommand{\ol}{\overline}
\newcommand{\oramr}[1]{\overrightarrow{\mathrm{#1}}}
\newcommand{\tri}{\triangle}
\newcommand{\mr}{\mathrm}
\newcommand{\mb}{\mathbb}
\newcommand{\mrvec}[1]{\overrightarrow{\mathrm{#1}}}
\newcommand{\itvec}{\overrightarrow}
\newcommand{\bs}{\boldsymbol}
\newcommand{\ra}{\rightarrow}
\newcommand{\Ra}{\Rightarrow}
\newcommand{\lra}{\longrightarrow}
\newcommand{\Lra}{\Longrightarrow}
\newcommand{\la}{\leftarrow}
\newcommand{\La}{\Leftarrow}
\newcommand{\lla}{\longleftarrow}
\newcommand{\Lla}{\Longleftarrow}
\newcommand{\lr}{\leftrightarrow}
\newcommand{\llr}{\longleftrightarrow}
\newcommand{\Llr}{\Longleftrightarrow}
\renewcommand{\deg}{{}^\circ}
\newcommand{\phbox}{\fbox{\phantom{1\hspace{2em}}}}
\newcommand{\boxnum}[1]{\fbox{\phantom{\hspace{1em}}({#1})\phantom{\hspace{1em}}}}
\newcommand{\boxkana}[1]{\fbox{\phantom{\hspace{1em}}{#1}\phantom{\hspace{1em}}}}
\newcommand{\boxkm}[2]{\fbox{\, {#1}\phantom{\hspace{0.2em}} \,  {#2}}}
\newcommand{\hzw}{\hspace{1\zw}}

\renewcommand{\baselinestretch}{1.25}
\parindent=1\zw

%入92

\begin{document}
\noindent
\fbox{NewTH1-19} [大阪大2006]

ある惑星と惑星探査機が太陽を含む平面内で運動しているとき,探査機をいったん惑星に接近させ遠ざけることで,探査機の速さを増加させることができる.その方法について以下の問いに答えよ.惑星の質量を$M$,探査機の質量を$m$とし,探査機,惑星および太陽の大きさは無視できるものとする.距離$R$だけ離れた惑星と探査機の間にはたらく万有引力の大きさは$\dfrac{GMm}{R^2}$であり,無限遠からはかった万有引力による位置エネルギーは$-\dfrac{GMm}{R}$である.ここで,$G$は万有引力定数である.惑星の質量は探査機の質量に比べて十分に大きく,また探査機が太陽や他の惑星から受ける引力は無視できるものとする.さらに,探査機が惑星に接近してその後遠ざかる間,惑星の運動を近似的に等速直線運動とみなしてよいものとする.以下のIおよびIIでは惑星から見た探査機の運動を考える.IIIでは太陽から見た探査機の運動を考える.
\begin{enumerate}[I]
  \item {\hzw}まず,惑星から見た探査機の運動を考えよう.探査機が惑星に最も接近したときの,惑星からの距離を$r_\mr{P}$,惑星に対する探査機の相対速度の大きさを$v_\mr{P}$とする.探査機が惑星を中心とする半径$r_\mr{P}$の等速円運動をするときの$v_\mr{P}$を$v_\mr{P} = v_1$とする.また,ある$v_2$に対して$v_\mr{P} \geqq v_2$となる場合には,探査機は惑星の引力の影響を脱して無限遠に到達できる.$v_1 < v_\mr{P} < v_2$となる場合には,探査機の軌跡は楕円を描く.
  \begin{enumerate}[(1)]
    \item {\hzw}$v_1$および$v_2$を$G$,$M$,$m$,$r_\mr{P}$のうち必要なものを用いて表せ.
  \end{enumerate}
  \item {\hzw}これ以降は,$v_\mr{P} > v_2$である場合を考えよう.この場合は,惑星から見た探査機の運動の軌跡は双曲線となる.このとき,図1のように適当な$x$軸,$y$軸および原点を選ぶと,この双曲線の$x$座標および$y$座標の間には,
  \begin{gather*}
    \frac{x^2}{a^2} - \frac{y^2}{b^2} = 1 \, (x > 0)
  \end{gather*}
  なる関係が成り立つ.ここで,$a > 0$,$b > 0$とする.
  探査機は無限遠で相対速度の大きさ$v_0$をもち漸近線$y = -\dfrac{b}{a}x$に沿って惑星に接近し,漸近線$y = \dfrac{b}{a}x$に沿って無限遠に遠ざかる.これらの漸近線と$x$軸がなす角度を$\theta\, \text{〔rad〕}$とする.ただし,$0 < \theta <\dfrac{\pi}{2}$である.惑星は$x$軸上の点Pに位置し,座標の原点Oと点Pとの距離は$\mr{OP} = \sqrt{a^2 + b^2}$となる.
  また,惑星と各漸近線との距離は$b$である,
  距離$b$と相対速度の大きさ$v_0$が与えられると,角度$\theta$(あるいは$a$)が定まり探査機の軌道を決定することができる.惑星と探査機を結ぶ線分が単位時間に通過する面積を面積速度とよぶ.図2に示すように,探査機の相対速度の大きさが$u$であるとき,面積速度の値は斜線で示した底辺の長さ$u$,高さ$h$の三角形の面積に等しく,$\dfrac{uh}{2}$となる.
  探査機が軌道上を運動する間,面積速度は一定に保たれる.
  \begin{enumerate}[(1), resume]
    \item {\hzw}探査機が無限遠から漸近線$y = -\dfrac{b}{a}x$に沿って惑星に接近するときの面積速度を$G$,$M$,$m$,$b$,$v_0$のうち必要なものを用いて表せ.また,探査機が惑星に最も接近したときの,探査機の相対速度の大きさ$v_\mr{P}$を$G$,$M$,$m$,$r_\mr{P}$,$b$,$v_0$のうち必要なものを用いて表せ.
    \item {\hzw}距離$r_\mr{P}$を$G$,$M$,$m$,$b$,$v_0$のうち必要なものを用いて表せ.
    \item {\hzw}角度$\theta$の正接関数$\tan\theta$を$G$,$M$,$m$,$b$,$v_0$のうち必要なものを用いて表せ.
    \item {\hzw}探査機が惑星に接近して遠ざかる間に,惑星が探査機に与えた力積の$x$成分および$y$成分を$G$,$M$,$m$,$b$,$v_0$,$\theta$のうち必要なものを用いて表せ.ただし,力積の$x$成分,$y$成分の正の向きをそれぞれ$x$軸,$y$軸の正の向きにとるものとする. 
  \end{enumerate}
  \item {\hzw}次に,太陽から見たIIの探査機の運動を考えよう.
  太陽から見た惑星の運動は,近似的に速さ$V$の等速直線運動とみなせる.図3に示すように,この直線運動は,図1の$x$軸から反時計回りに角度$\phi \, \text{〔rad〕}$の方向に向いているものとする.
  ただし,$-\pi \leqq \phi <\pi$である.
  \begin{enumerate}[(1), resume]
    \item {\hzw}太陽から見た探査機の運動を考えると,その運動エネルギーは,惑星の引力を受ける前に比べて,引力の影響を脱した後ではどれだけ変化するか.$G$,$M$,$m$,$V$,$v_0$,$\theta$,$\phi$のうち必要なものを用いて表せ.また,引力の影響を脱した後の探査機の運動エネルギーが引力の影響を受ける前に比べて増加するために,角度$\phi$が満たすべき条件を記せ.
  \end{enumerate}
\end{enumerate}


\end{document}
